\newcommand{\matr}[1]{\mathbf{#1}}
\renewcommand{\vec}[1]{\mathbf{#1}}
\newcommand{\spann}[1]{\mathrm{span}\{#1\}}
\newcommand{\transpose}{^\intercal}
\newcommand{\norm}[1]{\left\lVert#1\right\rVert}
\newcommand{\cond}[1]{\mathrm{cond}(#1)}
\newcommand{\bigO}{\mathcal{O}}

% List of all Abbreviations and Symbols
\prefacesection{Notation, Definitions, and Abbreviations}

\section*{Notation}
\begin{description}[font=\rmfamily\bfseries, leftmargin=3cm, style=nextline]
\item[\(\matr{A}, \vec{b}\)] Matrices and vectors are in math bold face.
\item[\(\matr{A}_i, \vec{b}_i\)] Subscript \(i\) denotes ``the \(i\)th matrix/vector''.
\item[\(a_{i,j}, b_{i}\)] Lowercase letters with subscript denotes ``the element at
  row \(i\) column \(j\) in matrix \(\matr{A}\)'' or ``the \(i\)th element in
  vector \(\vec{b}\)''.
\end{description}

% \section*{Definitions}
% \begin{description}[font=\rmfamily\bfseries, leftmargin=3cm, style=nextline]
% \item[Challenge] With respect to video games, a challenge is a set of goals
%   presented to the player that they are tasks with completing; challenges can
%   test a variety of player skills, including accuracy, logical reasoning, and
%   creative problem solving
% \end{description}

\section*{Abbreviations}
\begin{description}[font=\rmfamily\bfseries, leftmargin=3cm, style=nextline]
\item[GMRES] Generalized minimal residual method
\item[IC] Incomplete Cholesky factorization
\item[ILU] Incomplete LU factorization
\item[IR] Iterative refinement
\item[MGS] Modified Gram-Schmidt orthogonalization
\item[MP] Mixed-precision
\item[SPD] Symmetric positive definite
\end{description}
