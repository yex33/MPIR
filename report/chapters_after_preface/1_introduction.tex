\chapter{Introduction}
\label{cha:introduction}

In numerical computing, solving large sparse systems of linear equations
efficiently and accurately is a fundamental challenge with wide applications in
scientific computing, optimization, and engineering simulations. Traditional
double precision direct solvers, though robust, often suffer from high memory
demands and computational cost, particularly due to fill-in during factorization
of sparse matrices. This has motivated the development of iterative methods and
preconditioning strategies that aim to balance efficiency and accuracy.
Concretely, we are concerned with solving linear systems of the form \[\matr{A}
  \vec{x} = \vec{b}, \] where \(\matr{A} \in \mathbb{R}^{n \times n}\) is a
symmetric, nonsingular, sparse matrix and \(\vec{b} \in \mathbb{R}^n\) is a
specified right-hand side vector.

One promising approach is mixed-precision iterative refinement (IR), which
leverages lower precision arithmetic for speed while retaining high precision
steps to ensure stability and accuracy. In particular, the GMRES-based iterative
refinement (GMRES-IR) method combines iterative refinement with the Generalized
Minimal Residual method (GMRES), enabling robust convergence even for
ill-conditioned systems \cite{lindquist_improving_2020,mary_mixed_2023}.
Mixed-precision algorithms have been shown to significantly reduce the
time-to-solution by exploiting modern hardware accelerators and memory
hierarchies, while maintaining accuracy comparable to traditional
double-precision solvers \cite{mary_mixed_2023}.

This project builds on prior work by \textcite{wong_exploring_2024}, who developed
a prototype GMRES-IR solver with theoretical analysis in a five-precision
setting. Building on this foundation, the present work explores and experiments
with a practical implementation of GMRES-IR in C++, emphasizing both algorithmic
advances and software design improvements. The solver supports mixed precision
across different stages of the refinement process, enforced through C++
templates and concepts that explicitly separate factorization, working, and
residual precisions. Compared to the earlier implementation, the codebase has
been modernized with a cleaner interface, integrated CMake build integration,
and improved modularity.

A key obstacle in earlier GMRES-IR implementations was the reliance on
LDL\textsuperscript{T} factorization (via a modified QDLDL package
\cite{shahrooz_derakhshan_using_2023}) for preconditioning, which incurred
substantial fill-in and dominated the runtime. To address this, we replace
LDL\textsuperscript{T} with a fine-grained parallel incomplete LU (ILU)
factorization \cite{chow_fine-grained_2015}, which provides a less accurate but
effective preconditioner with significantly reduced memory overhead.
Implementing ILU in parallel within a mixed-precision framework requires careful
design but enables scalability to much larger sparse problems. In the present
implementation, ILU is applied only to symmetric matrices. Most of these are
symmetric positive definite (SPD), though some are indefinite. While ILU is, in
principle, applicable to general nonsymmetric systems, robust application in
that setting typically requires reordering or pivoting to ensure stability,
which is beyond the scope of this project. Restricting attention to symmetric
(mostly SPD) matrices keeps the implementation tractable while still allowing
evaluation of the effectiveness of a mixed-precision GMRES-IR solver with
parallel ILU preconditioning.

The main contribution of this report is an experimental study of the performance
trade-offs in mixed-precision GMRES-IR with parallel ILU preconditioning. We
investigate the effects of precision choice, Krylov subspace size, and
preconditioning quality on both runtime and accuracy when solving large sparse
systems. While the implementation is not yet a polished solver intended for
production use, it provides insight into practical aspects of mixed-precision
iterative refinement and contributes to the broader goal of developing robust,
efficient open-source solvers.

The remainder of this report is structured as follows:

\begin{itemize}
\item Chapter \ref{cha:literature-review} reviews the relevant literature, covering
  the foundations of iterative solvers, preconditioning techniques, and
  mixed-precision computing.
\item Chapter \ref{cha:math-form} presents the mathematical formulation of the core
  algorithms, including the GMRES method, the iterative refinement framework,
  and the fine-grained parallel ILU factorization.
\item The C++ implementation of the solver is described in Chapter
  \ref{cha:implementation}, with a focus on its software architecture, use of
  templates to ensure precision flexibility, and modular design.
\item In Chapter \ref{cha:exper-eval}, we present a comprehensive experimental
  evaluation, assessing the solver’s accuracy and performance across a benchmark
  of sparse matrices and analyzing the trade-offs of various mixed-precision
  configurations.
\item Finally, Chapter \ref{cha:conclusion} concludes the report with a summary of our
  findings and offers directions for future research.
\end{itemize}
