\chapter{Experiment and Evaluation}
\label{cha:exper-eval}

\section{Experimental Setup}
\label{sec:experimental-setup}

All experiments were conducted on a Intel\textsuperscript{\textregistered}
Core\textsuperscript{\texttrademark} i7-10700K CPU, with 8 physical cores (16
threads), base frequency 3.8 GHz and turbo frequency up to 4.7 GHz. The system
was equipped with 32 GB DDR4 RAM.

\section{Test Matrices}
\label{sec:test-matrices}

All test problems were drawn from the Florida Sparse Matrix Collection
\cite{davis_university_2011}, restricted to symmetric positive definite
matrices. Table~\ref{tab:test-matrices} summarizes the benchmark set, ordered by
increasing condition number.

Each matrix \(\matr{A}\) was read from a \texttt{.mtx} file using the
\texttt{fast\_matrix\_market}\footnote{\href{https://github.com/alugowski/fast_matrix_market}{\texttt{fast\_matrix\_market}:
    fast and full-featured Matrix Market I/O library for C++, Python, and R}}
library. The right-hand side \(\vec{b}\) was constructed consistently across all
experiments from a reference solution \[\vec{x}_{\text{ref}} = (1, 1, \dots,
  1)\transpose \in \mathbb{R}^{n}, \quad \vec{b} =
  \matr{A}\vec{x}_{\text{ref}}.\] This guarantees that the exact solution is
known, enabling direct computation of both forward and backward errors.

\begin{table}[h]
  \centering
  \begin{tabular}{
    l                                              % Matrix
    S[table-format=1.1e2]                          % condest(A)
    S[table-format=1.1e1]                          % size
    S[table-format=1.1e1]                          % NNZ
    S[table-format=1.4, print-zero-exponent=false] % % NZZ
    S[table-format=1.1e3]                          % min
    S[table-format=1.1e3]                          % max
    }
    \toprule
    Matrix                            & {\texttt{condest(A)}} & {Size}  & {\texttt{nnz(A)}} & {\% NNZ} & {\texttt{min(A)}} & {\texttt{max(A)}} \\
    \midrule
    {\footnotesize \texttt{crystm01}}       & 4.2e+02         & 4.9e+03 & 1.1e+05     & 0.4      & 7.5e-15     & 1.8e-12     \\
    {\footnotesize \texttt{Dubcova3}}       & 1.1e+04         & 1.5e+05 & 3.6e+06     & 0.02     & 8.5e-22     & 2.7e+00     \\
    {\footnotesize \texttt{bcsstk09}}       & 3.1e+04         & 1.1e+03 & 1.8e+04     & 2        & 1.1e-08     & 3.9e+07     \\
    {\footnotesize \texttt{parabolic\_fem}} & 2.1e+05         & 5.3e+05 & 3.7e+06     & 0.001    & 3.2e-07     & 4.0e-01     \\
    {\footnotesize \texttt{494\_bus}}       & 3.9e+06         & 4.9e+02 & 1.7e+03     & 0.7      & 1.7e-01     & 2.0e+04     \\
    {\footnotesize \texttt{cfd2}}           & 4.2e+06         & 1.2e+05 & 3.1e+06     & 0.02     & 6.7e-09     & 1.0e+00     \\
    {\footnotesize \texttt{lund\_a}}        & 5.4e+06         & 1.5e+02 & 2.4e+03     & 1e+01    & 1.2e-04     & 1.5e+08     \\
    {\footnotesize \texttt{1138\_bus}}      & 1.2e+07         & 1.1e+03 & 4.1e+03     & 0.3      & 4.8e-01     & 2.0e+04     \\
    {\footnotesize \texttt{G3\_circuit}}    & 2.2e+07         & 1.6e+06 & 7.7e+06     & 0.0003   & 3.3e-01     & 2.3e+04     \\
    {\footnotesize \texttt{bcsstk08}}       & 4.7e+07         & 1.1e+03 & 1.3e+04     & 1        & 1.8e-12     & 7.6e+10     \\
    {\footnotesize \texttt{ecology2}}       & 6.7e+07         & 1.0e+06 & 5.0e+06     & 0.0005   & 1.0e+00     & 4.0e+01     \\
    {\footnotesize \texttt{cbuckle}}        & 8.0e+07         & 1.4e+04 & 6.8e+05     & 0.4      & 3.9e-31     & 3.7e+04     \\
    {\footnotesize \texttt{2cubes\_sphere}} & 2.9e+09         & 1.0e+05 & 1.6e+06     & 0.02     & 6.7e-15     & 2.5e+10     \\
    {\footnotesize \texttt{bcsstk14}}       & 1.3e+10         & 1.8e+03 & 6.3e+04     & 2        & 1.6e-27     & 8.9e+09     \\
    {\footnotesize \texttt{bcsstk18}}       & 6.5e+11         & 1.2e+04 & 1.5e+05     & 0.1      & 2.1e-25     & 3.1e+10     \\
    {\footnotesize \texttt{plat362}}        & 7.1e+11         & 3.6e+02 & 5.8e+03     & 4        & 3.5e-21     & 4.6e-01     \\
    {\footnotesize \texttt{offshore}}       & 2.3e+13         & 2.6e+05 & 4.2e+06     & 0.006    & 7.2e-21     & 7.5e+14     \\
    {\footnotesize \texttt{ct20stif}}       & 2.2e+14         & 5.2e+04 & 2.6e+06     & 0.09     & 3.0e-34     & 8.9e+11     \\
    {\footnotesize \texttt{bloweybq}}       & 4.2e+18         & 1.0e+04 & 5.0e+04     & 0.05     & 2.5e-01     & 5.0e+03     \\
    \bottomrule
  \end{tabular}
  \caption[SPD matrices used for experiments]{SPD matrices used for experiments.
    \texttt{condest(A)} is the lower bound of the 1-norm condition number of
    matrix \(\matr{A}\) computed by MATLAB\textsuperscript{\textregistered}.}
  \label{tab:test-matrices}
\end{table}

\section{Mixed-Precision Notation and Configurations}
\label{sec:mixed-prec-notat}

Throughout our experiments we adopt three mixed-precision configurations,
denoted as SDD, DDD, and DDQ. These notations indicate the arithmetic precision
chosen for the three computational components: factorization \((u_f)\), working
\((u_w)\), and residual evaluation \((u_r)\). Table~\ref{tab:notation}
summarizes the mapping.

\begin{table}[h]
  \centering
  \begin{tabular}{lccc}
    \toprule
    Configuration & \(u_f\)    & \(u_w\) ​   & \(u_r\)       \\
    \midrule
    SDD           & \texttt{single} & \texttt{double} & \texttt{double}    \\
    DDD           & \texttt{double} & \texttt{double} & \texttt{double}    \\
    DDQ           & \texttt{double} & \texttt{double} & \texttt{quadruple} \\
    \bottomrule
  \end{tabular}
  \caption[Precision configurations for mixed-precision IR]{Notation for the three
    precision configurations tested in this work. \(u_f\) corresponds to the
    factorization precision, \(u_w\) to the GMRES working precision, and \(u_r\)
    to the residual evaluation precision.}
  \label{tab:notation}
\end{table}


\section{Residual and Error Metrics}
\label{sec:resid-error-metr}

For iterative refinement and GMRES, two measures are reported:
\begin{enumerate}
\item relative residual (backward error), given by \[\epsilon_{\text{res}} =
    \frac{\norm{\vec{b} - \matr{A} \vec{x}_k}_{\infty}}{\norm{\vec{b}}_{\infty}},\] where \(\vec{x}_k\) is the approximate solution
  after \(k\) iterations;
\item relative error (forward error), given by \[\epsilon_{\text{err}} = \frac{\norm{\vec{x}_k -
        \vec{x}_{\text{ref}}}_{\infty}}{\norm{\vec{x}_{\text{ref}}}_{\infty}}.\]
\end{enumerate}

To assess the quality of the preconditioner, the same non-linear residual norm
used in \cite{chow_fine-grained_2015} is also monitored: \[ \sum_{(i,j) \in S}
  \left| a_{i,j} - \sum_{k=1}^{\min(i,j)} l_{i,k}u_{k,j} \right| \]

\section{Convergence of ILU}
\label{sec:convergence-ilu}

For the experiments in this section, a chunk size of 64 was used, meaning each
thread processed 64 columns per batch. This choice follows the column-to-matrix
size ratio used in \cite{chow_fine-grained_2015}.

The mixed-precision configuration employed was DDQ, i.e., factorization in
\texttt{double} precision, GMRES iterations in \texttt{double} precision, and
residual computation in \texttt{octuple} precision. The target tolerance was set
to \num{1e-18}, which is close to the machine epsilon of \texttt{double}
precision, in order to stress the iterative refinement procedure and push the
working precision toward its limits.

The outer iteration count for IR was capped at 200, while the maximum Krylov
subspace dimension for GMRES (inner iteration) was capped at 50. These bounds
were chosen to maintain consistency with the other experiments reported in this
chapter.

Table~\ref{tab:sweeps} reports the nonlinear residual norm at each sweep for the
case of ILU(0) for matrix \texttt{bcsstk18}. The values in parentheses indicate
the total number of IR iterations performed before convergence under the given
configuration. Results are shown for varying thread counts and number of sweeps
performed.
\begin{table}[h]
  \centering
  \newrobustcmd\iter[1]{%
    (#1)
  }
  \sisetup{table-format = 1.1e4 \iter{a}}
  \begin{tabular}{
    c
    S % 1
    S % 2
    S % 3
    S % 4
    S % 5
    }
    \toprule
    Thread & \multicolumn{5}{c}{Sweeps (Iterations)}                                                                    \\
           & \text{1}          & \text{2}          & \text{3}          & \text{4}          & \text{5}          \\
    \midrule
    1      & 4.0e-13 \iter{32} & 4.0e-13 \iter{32} & 4.0e-13 \iter{32} & 4.0e-13 \iter{32} & 4.0e-13 \iter{32} \\
    2      & 4.9e+01 \iter{36} & 5.9e-01 \iter{32} & 1.5e-03 \iter{34} & 3.7e-06 \iter{32} & 1.7e-08 \iter{32} \\
    4      & 8.9e+01 \iter{38} & 1.5e+00 \iter{32} & 5.0e-02 \iter{34} & 3.9e-03 \iter{32} & 2.3e-05 \iter{32} \\
    8      & 1.3e+02 \iter{41} & 3.4e+00 \iter{34} & 1.9e-01 \iter{34} & 1.4e-02 \iter{31} & 3.9e-04 \iter{32} \\
    16     & 1.9e+02 \iter{36} & 6.0e+00 \iter{30} & 3.7e-01 \iter{32} & 3.7e-02 \iter{32} & 2.3e-03 \iter{32} \\
    \bottomrule
  \end{tabular}
  \caption[Nonlinear residual norms for matrix \texttt{bcsstk18}]{Nonlinear residual
    norms for matrix \texttt{bcsstk18} across sweeps and thread counts. Values
    in parentheses denote the number of IR iterations required for convergence.}
  \label{tab:sweeps}
\end{table}

From these results, it is evident that the number of IR iterations required for
convergence stabilizes after approximately 5 sweeps, regardless of the thread
count. Consequently, all subsequent experiments are conducted with the maximum
number of sweeps set to 5.

\section{Accuracy}
\label{sec:accuracy}

Table~\ref{tab:res-accuracy} reports the relative forward error and relative
backward error for each test matrix across three mixed-precision settings: SDD,
DDD, and DDQ. The IR outer iteration count was capped at 200, and the Krylov
subspace dimension for GMRES (inner iteration) was limited to 50. Level 0 ILU
preconditioning was performed with up to 5 sweeps, executed in parallel with 16
threads. A tolerance of \(\num{1e-18}\) was used throughout. Entries marked with
an asterisk indicate cases where the IR process reached the maximum of 200
iterations without convergence, and the reported residual corresponds to the
final iteration.

\begin{table}[h]
  \centering
  \newrobustcmd\iter[1]{(#1)}
  \begin{threeparttable}
    \begin{tabular}{ l S[table-format = 1.1e3] % SDD
      S[table-format = 1.1e3] % DDD
      S[table-format = 1.1e3] % DDQ
      | S[table-format = 1.1e4 \iter{12}] % SDD
      S[table-format = 1.1e4 \iter{12}] % DDD
      S[table-format = 1.1e4 \iter{12}] % DDQ
      }
      \toprule
      Matrix \tnote{\(\dagger\)}                      & \multicolumn{3}{c}{Relative Error}            & \multicolumn{3}{c}{Relative Residual (Iterations)}                  \\
      \midrule
                                   & \text{SDD} & \text{DDD} & \text{DDQ} & \text{SDD}         & \text{DDD}         & \text{DDQ}         \\
      {\footnotesize \texttt{1138\_bus}} & 9.3e-12    & 8.9e-12    & 6.6e-12    & 6.2e-15 \iter{187} & 3.7e-15 \iter{43}  & 2.5e-15 \iter{48}  \\
      {\footnotesize \texttt{2cubes\_s}} & 1.8e-15    & 1.8e-15    & 1.3e-15    & 3.2e-16 \iter{2}   & 3.2e-16 \iter{2}   & 2.1e-16 \iter{2}   \\
      {\footnotesize \texttt{494\_bus}}  & 2.6e-12    & 2.0e-12    & 1.5e-12    & 1.7e-15 \iter{47}  & 1.7e-15 \iter{94}  & 3.3e-15 \iter{16}  \\
      {\footnotesize \texttt{Dubcova3}}  & 1.4e-13    & 1.4e-13    & 1.7e-13    & 7.9e-16 \iter{9}   & 7.9e-16 \iter{9}   & 5.3e-16 \iter{10}  \\
      {\footnotesize \texttt{G3\_circu}} & 1.0e-11    & 2.7e-11    & 4.8e-12    & 5.0e-16 \iter{63}  & 5.0e-16 \iter{60}  & 5.0e-16 \iter{64}  \\
      {\footnotesize \texttt{bcsstk08}}  & 3.0e-12    & 1.1e-12    & 6.7e-12    & 4.0e-15 \iter{2}   & 2.1e-15 \iter{19}  & 6.3e-16 \iter{2}   \\
      {\footnotesize \texttt{bcsstk09} \tnote{\(\ddag\)}}  & 1.3e+00    & 1.6e+00    & 1.6e+00    & 5.3e-01 \iter{*}   & 6.1e-01 \iter{*}   & 6.0e-01 \iter{*}   \\
      {\footnotesize \texttt{bcsstk14}}  & 2.3e-08    & 3.1e-09    & 2.5e-09    & 1.2e-12 \iter{*}   & 1.6e-13 \iter{*}   & 1.3e-13 \iter{*}   \\
      {\footnotesize \texttt{bcsstk18}}  & 7.7e-11    & 8.6e-11    & 8.3e-11    & 4.9e-16 \iter{30}  & 3.9e-16 \iter{34}  & 3.0e-16 \iter{32}  \\
      {\footnotesize \texttt{bloweybq}}  & 1.1e+00    & 1.5e-01    & 1.6e-01    & 6.1e-15 \iter{*}   & 2.9e-19 \iter{2}   & 3.3e-19 \iter{4}   \\
      {\footnotesize \texttt{cbuckle}}   & 2.0e-12    & 2.8e-12    & 1.7e-12    & 2.2e-15 \iter{49}  & 2.7e-15 \iter{9}   & 1.7e-15 \iter{7}   \\
      {\footnotesize \texttt{crystm01}}  & 4.9e-15    & 5.1e-15    & 3.9e-15    & 5.4e-16 \iter{3}   & 6.7e-16 \iter{4}   & 4.0e-16 \iter{2}   \\
      {\footnotesize \texttt{ct20stif} \tnote{\(\ddag\)}}  & 1.0e+00    & 7.2e+00    & 7.1e+00    & 1.0e+00 \iter{1}   & 3.4e-08 \iter{*}   & 3.2e-08 \iter{*}   \\
      {\footnotesize \texttt{ecology2}}  & 7.3e-01    & 7.3e-01    & 7.1e-01    & 2.6e-06 \iter{*}   & 2.4e-06 \iter{*}   & 1.9e-05 \iter{*}   \\
      {\footnotesize \texttt{lund\_a}}   & 3.6e-12    & 2.9e-12    & 8.3e-13    & 6.2e-16 \iter{4}   & 5.0e-16 \iter{2}   & 5.0e-16 \iter{2}   \\
      {\footnotesize \texttt{offshore}}  & 8.7e-10    & 1.0e-08    & 3.0e-09    & 4.3e-16 \iter{18}  & 4.3e-16 \iter{16}  & 3.8e-16 \iter{17}  \\
      {\footnotesize \texttt{paraboli}}  & 1.1e-11    & 1.1e-11    & 4.1e-12    & 7.4e-11 \iter{151} & 7.3e-11 \iter{115} & 5.8e-11 \iter{124} \\
      {\footnotesize \texttt{plat362}}   & 4.8e+00    & 4.4e+00    & 4.4e+00    & 6.7e-04 \iter{199} & 5.1e-04 \iter{*}   & 5.1e-04 \iter{*}   \\
      \bottomrule
    \end{tabular}
    \caption[Relative forward error and relative residual]{Relative forward
      error and relative residual for test matrices under three MP
      configurations. Rows are sorted by matrix name. Residuals are shown with
      the number of IR outer iterations in parentheses. Asterisks indicate cases
      where the maximum IR iteration limit was reached.}
    \label{tab:res-accuracy}

    \begin{tablenotes}
    \item[\(\dagger\)] Matrix identifiers have been truncated to 8 characters (see
      Table~\ref{tab:test-matrices} for full names).
    \item[\(\ddag\)] ILU factorization diverged.
    \end{tablenotes}
  \end{threeparttable}
\end{table}

\subsection{Discussion}

The results in Table~\ref{tab:res-accuracy} highlight several key aspects of
accuracy behavior across the mixed-precision configurations. First, since the
reference solution used to compute the relative forward error is exact, the
relative error is consistently larger than the relative residual. This
discrepancy generally aligns with the condition number of the tested matrix.

Across most test cases, the relative error and residual are consistent across
the three MP configurations (SDD, DDD, DDQ). A slight trend of improved accuracy
is observed as higher precision is employed, though the reduction is
modest---typically within one order of magnitude. Notably, for extremely
ill-conditioned problems such as \texttt{bloweybq}, the shift from single to
double precision in factorization yields a significant improvement in relative
residual reduction.

The iteration counts further illustrate the effect of precision change. Moving
from \texttt{single} to \texttt{double} in factorization generally reduces the
number of IR iterations required for convergence. Increasing the residual
precision from \texttt{double} to \texttt{octuple} slightly increases the
iteration count in many cases, despite achieving comparable or better accuracy.
This suggests that higher factorization generally accelerates convergence, while
higher residual precision stabilizes the residual but does not necessarily
accelerate convergence.

For the problematic cases marked with asterisks, different convergence behaviors
were observed. For \texttt{bcsstk14} and \texttt{plat362}, the method showed
steady progress but failed to converge within the iteration limit, indicating
that additional IR iterations could lead to eventual convergence. In contrast,
for \texttt{ecology2}, the residual stagnated entirely, suggesting that
increasing the Krylov subspace dimension or employing a more robust
preconditioner is necessary. For \texttt{bcsstk09} and \texttt{ct20stif}, the
ILU(0) preconditioner diverged, which in turn rendered the subsequent GMRES
solve nearly unusable and resulted in large forward errors and residuals.
Interestingly, in the case of \texttt{bcsstk09}, a development run of GMRES with
the preconditioner manually removed converged to a substantially lower residual,
indicating that the breakdown originates in the ILU factorization itself rather
than in the Krylov solver. \textcite{chow_fine-grained_2015} provides a
theoretical guarantee of convergence, which stands in contrast with the
experimental failures observed here. The inconsistency suggests that the
divergence is likely due to a flaw in our current ILU implementation rather than
a fundamental limitation of the method. Multiple attempts were made to isolate
the source of the issue, but we were unable to definitively identify the cause
within the scope of this work.

\section{Timing}
\label{sec:timing}

Table~\ref{tab:timing} summarizes the factorization and solve times measured for
the same set of tests reported in Section~\ref{sec:accuracy}. Factorization time
refers to the ILU factorization, while solve time includes the cost of iterative
refinement, GMRES iterations, and residual evaluations. As before, an asterisk
denotes cases where the solver reached the maximum of 200 IR iterations, and the
timing corresponds to the capped run.

\begin{table}[h]
  \centering
  \newrobustcmd\iter[1]{#1}
  \sisetup{
    table-format = 3.3,
    print-zero-exponent = false,
    table-align-text-after = false
  }
  \begin{tabular}{
    l
    S % SDD
    S % DDD
    S % DDQ
    |
    S % SDD
    S % DDD
    S % DDQ
    }
    \toprule
    Matrix                            & \multicolumn{3}{c}{Factorization Time (s)}    & \multicolumn{3}{c}{Solve Time (s)}                          \\
    \midrule
                                      & \text{SDD} & \text{DDD} & \text{DDQ} & \text{SDD}      & \text{DDD}      & \text{DDQ}       \\
    {\footnotesize \texttt{1138\_bus}}      & 0.02       & 0.01       & 0.01       & 0.59            & 0.15            & 0.19             \\
    {\footnotesize \texttt{2cubes\_sphere}} & 0.50       & 0.48       & 0.48       & 1.59            & 1.62            & 2.70             \\
    {\footnotesize \texttt{494\_bus}}       & 0.01       & 0.01       & 0.01       & 0.06            & 0.12            & 0.03             \\
    {\footnotesize \texttt{Dubcova3}}       & 10.94      & 11.96      & 11.87      & 15.95           & 17.41           & 27.67            \\
    {\footnotesize \texttt{G3\_circuit}}    & 0.42       & 0.47       & 0.46       & 588.19          & 566.95          & 711.52           \\
    {\footnotesize \texttt{bcsstk08}}       & 0.02       & 0.01       & 0.01       & 0.01            & 0.07            & 0.03             \\
    {\footnotesize \texttt{bcsstk09}}       & 0.03       & 0.01       & 0.03       & 0.69 \iter{*}   & 1.04 \iter{*}   & 1.91 \iter{*}    \\
    {\footnotesize \texttt{bcsstk14}}       & 0.03       & 0.02       & 0.02       & 1.87 \iter{*}   & 1.92 \iter{*}   & 4.84 \iter{*}    \\
    {\footnotesize \texttt{bcsstk18}}       & 0.05       & 0.09       & 0.05       & 1.43            & 1.69            & 2.65             \\
    {\footnotesize \texttt{bloweybq}}       & 0.03       & 0.04       & 0.03       & 5.45 \iter{*}   & 0.09            & 0.23             \\
    {\footnotesize \texttt{cbuckle}}        & 0.22       & 0.24       & 0.23       & 5.21            & 1.08            & 2.03             \\
    {\footnotesize \texttt{crystm01}}       & 0.04       & 0.03       & 0.04       & 0.08            & 0.08            & 0.09             \\
    {\footnotesize \texttt{ct20stif}}       & 0.77       & 1.72       & 3.40       & 0.01            & 231.85 \iter{*} & 461.94 \iter{*}  \\
    {\footnotesize \texttt{ecology2}}       & 0.55       & 0.44       & 0.45       & 899.84 \iter{*} & 908.84 \iter{*} & 1132.97 \iter{*} \\
    {\footnotesize \texttt{lund\_a}}        & 0.02       & 0.01       & 0.01       & 0.00            & 0.00            & 0.01             \\
    {\footnotesize \texttt{offshore}}       & 1.24       & 1.25       & 1.27       & 25.90           & 23.64           & 42.34            \\
    {\footnotesize \texttt{parabolic\_fem}} & 0.20       & 0.23       & 0.23       & 318.05          & 246.96          & 372.64           \\
    {\footnotesize \texttt{plat362}}        & 0.01       & 0.02       & 0.01       & 0.25            & 0.25 \iter{*}   & 0.53 \iter{*}    \\
    \bottomrule
  \end{tabular}
  \caption[Runtime performance for mixed-precision IR]{Factorization and solve
    times (in seconds) for test matrices under three precision configurations:
    SDD, DDD, and DDQ. Rows are sorted by matrix name. Asterisks indicate runs
    that reached the IR iteration cap of 200.}
  \label{tab:timing}
\end{table}

\subsection{Timing Discussion}

The results in Table~\ref{tab:timing} show that, across all test matrices, the
solve phase is consistently far more expensive than the factorization phase.
This stands in contrast to the findings of \textcite{wong_exploring_2024}, where
factorization dominated the runtime. The discrepancy arises primarily from
methodological differences: while \citeauthor{wong_exploring_2024} employed a
complete LDL\textsuperscript{T} factorization, we rely on an incomplete
factorization (ILU) which has substantially less unknowns to compute. In our
implementation the factorization is additionally parallelized across multiple
threads. The GMRES iterations that dominate the solve phase are limited to
sequential execution, thereby shifting the runtime heavily toward the solve
stage.

Another observation is that the time gain from performing factorization in
\texttt{single} precision versus \texttt{double} precision is minimal. This
outcome is consistent with the small contribution of factorization time relative
to the overall runtime. As solve time dominates, reducing factorization cost has
little impact on total runtime.

Finally, improving the residual computation from \texttt{double} precision to
\texttt{quadruple} precision results in a marked increase in runtime. This
slowdown arises from two sources. First, \texttt{quadruple} precision arithmetic
is intrinsically more expensive than \texttt{double} precision. Second,
higher-precision residuals demand additional GMRES iterations to converge as
observed in Section \ref{sec:accuracy}, further amplifying the computational
burden. Together, these factors explain the consistent timing increase observed
in the DDQ configuration compared to SDD and DDD.
