\chapter{Conclusion}
\label{cha:conclusion}

This work presented the mathematical foundations, software design,
implementation, and evaluation of a mixed-precision iterative refinement solver
for large sparse linear systems, combining a fine-grained parallel ILU
factorization with a preconditioned GMRES kernel. The solver was implemented in
modern C++ with emphasis on modularity, template-based precision flexibility,
and interface compatibility with existing linear algebra ecosystems. Extensions
to the QD library were introduced to enable seamless integration of quadruple
and octuple precision types within this framework.

The experimental results demonstrated that mixed precision can improve both
accuracy and robustness, though the gains are subtler than anticipated. Moving
from single to double precision in the factorization consistently accelerated
convergence and reduced iteration counts, while raising the residual precision
from double to quadruple stabilized the refinement process but significantly
increased runtime. Overall, the runtime advantage of lower precision
factorization was modest, since solve time was consistently dominated by GMRES
iterations.

Several pathological cases highlighted limitations of the current approach. For
some matrices, the \texttt{ILU(0)} preconditioner failed to converge, producing
stagnation or divergence in the subsequent GMRES solve. Although
\textcite{chow_fine-grained_2015} established convergence guarantees for the ILU
fixed-point sweeps used in factorization, these guarantees do not extend to the
full GMRES-IR solver. The observed breakdowns therefore point to weaknesses in
the present \texttt{ILU(0)} implementation and the inherent fragility of level-0
incomplete factorizations, rather than a fundamental flaw in the iterative
refinement framework itself.

The performance profile further emphasizes this distinction. Factorization was
consistently cheaper than the solve stage, in contrast to earlier LDLT-based
implementations where factorization dominated runtime. This shift arises from
the use of parallel \texttt{ILU(0)}, which reduces factorization cost, while
GMRES iterations remained sequential and therefore became the bottleneck.

\section*{Future Directions}

Two practical extensions follow directly from the limitations observed in this
work:

\begin{description}
\item[Parallel ILU for nonsymmetric systems] Extend the current fine-grained
  parallel ILU implementation beyond symmetric/SPD matrices to fully
  nonsymmetric cases. This would require stability safeguards such as diagonal
  scaling, threshold dropping, or optional pivoting/reordering, but would
  broaden applicability to a wider range of real-world problems such as
  convection-dominated PDEs and circuit models.

\item[Stronger ILU variants] Investigate \texttt{ILU(k)} and threshold-based \texttt{ILU(T)}
  factorizations within the same parallel framework. These approaches may
  provide more robust preconditioning for ill-conditioned systems where
  \texttt{ILU(0)} failed (e.g., \texttt{ct20stif}, \texttt{ecology2}), without
  sacrificing too much efficiency.
\end{description}

\section*{Closing Remarks}

In summary, this project developed and evaluated a modern C++ solver framework
for mixed-precision iterative refinement with parallel incomplete
factorizations. The results confirm that mixed precision can enhance solver
accuracy and robustness, but also show that the benefits depend critically on
the interplay between factorization quality, preconditioner reliability, and
Krylov convergence. While the immediate performance improvements were modest,
the implementation provides insights into precision hierarchies, highlights the
importance of preconditioner design, and establishes a foundation for future
advances in mixed-precision sparse linear solvers.
