\chapter{Conclusion}
\label{cha:conclusion}

This report presented the mathematical theories, design, implementation, and
evaluation of a mixed-precision iterative refinement solver for sparse linear
systems, with a focus on combining a fine-grained parallel incomplete LU
factorization with a preconditioned GMRES kernel. The solver was implemented in
modern C++ with a strong emphasis on modularity, precision flexibility, and
interface compatibility. Through a combination of template programming,
concept-driven type safety, and extensions to the QD library, the implementation
supports seamless use of both IEEE 754 and extended floating-point formats
across different solver components.

The numerical experiments demonstrated that mixed precision can deliver
improvements in both runtime and accuracy, though the extent of these gains is
sensitive to the interplay between factorization quality and Krylov convergence.
In particular, while moving from \texttt{single} to \texttt{double} precision in the
factorization consistently improved convergence rates, the timing advantage of
using lower-precision arithmetic in this solver architecture was modest.
Similarly, computing residuals in extended precision stabilized the refinement
process but incurred significant runtime penalties due to both slower arithmetic
and increased iteration counts. These results suggest that the benefits of mixed
precision in this setting are more nuanced than expected and depend heavily on
the balance of computational costs across the solver pipeline.

Several pathological cases revealed limitations in the current implementation.
For some test matrices, the \texttt{ILU(0)} preconditioner diverged, undermining
the effectiveness of the subsequent GMRES solve and leading to large forward
errors. This behavior stands in contradiction to existing theoretical guarantees
of ILU-based refinement, pointing toward flaws in the current implementation of
the factorization. Although repeated attempts were made to isolate the root
cause, a definitive resolution was not achieved within the scope of this work.
This underscores the inherent complexity of incomplete factorization algorithms
and their sensitivity to both numerical stability and implementation detail.

In terms of performance analysis, the factorization phase was consistently much
cheaper than the solve phase, which was dominated by sequential GMRES
iterations. This result contrasts with prior work using complete LDLT
factorizations, where factorization dominated runtime. The disparity highlights
the effect of both algorithmic choice (incomplete vs.\ complete factorization)
and parallelization strategy (parallel factorization vs.\ sequential Krylov
iterations). These findings reinforce the importance of considering the overall
solver pipeline when evaluating mixed-precision strategies.

\section*{Future Directions}
\label{sec:future-directions}

Several promising directions emerge from this work:

\begin{itemize}
\item Lower precision arithmetic on accelerators. While this work focused on
  \texttt{single}, \texttt{double}, and \texttt{quadruple} precision, future
  research could investigate half-precision or other lower-precision formats,
  particularly in the context of GPU acceleration. Such approaches could yield
  significant runtime improvements, but they also introduce challenges such as
  limited dynamic range, overflow and underflow risks, and numerical
  instability. Techniques such as scaling, iterative refinement safeguards, and
  early breakdown detection could help mitigate these issues
  \cite{scott_note_2024}.
\item More robust preconditioners. The divergence of \texttt{ILU(0)} on certain matrices
  suggests the need for more reliable preconditioning strategies. Potential
  avenues include higher-level \texttt{ILU(k)}, and \texttt{ILU(T)} with
  threshold-based dropping. Such alternatives may improve stability without
  sacrificing the parallelism and efficiency achieved in this work.
\item Expanded parallelism in Krylov solvers. Since GMRES iterations dominate the
  runtime and remain sequential in the current implementation, future work could
  explore, parallel Krylov methods reduce this bottleneck and better align with
  the parallelized factorization.
\end{itemize}

In summary, this thesis contributes a modern C++ solver framework for
mixed-precision iterative refinement with parallel incomplete factorizations,
demonstrates its performance and accuracy characteristics, and identifies key
challenges that motivate future advances. While the immediate gains from mixed
precision in this solver design were more modest than anticipated, the results
provide important insights into the conditions under which mixed precision is
effective, and they lay the groundwork for further exploration of precision
hierarchies, preconditioning robustness, and algorithmic parallelism in sparse
linear system solvers.
