%\documentclass[handout]{beamer}
\documentclass[t,12pt,numbers,fleqn]{beamer}
%\documentclass[ignorenonframetext]{beamer}

\newif\ifquestions
%\questionstrue
\questionsfalse

\usepackage{pgfpages}
\usepackage{hyperref}
\hypersetup{colorlinks=true,
    linkcolor=blue,
    citecolor=blue,
    filecolor=blue,
    urlcolor=blue,
    unicode=false}
\urlstyle{same}

\usepackage{booktabs}

\usepackage{caption}
\usepackage{subcaption}
\captionsetup[figure]{font=scriptsize,labelfont=scriptsize}

\usepackage[version=4]{mhchem}
\usepackage[print-unity-mantissa=false]{siunitx}
\usepackage{tabularx}

\usepackage{algpseudocode,algorithm}

\usepackage[backend=biber,style=authoryear]{biblatex}
\addbibresource{../../../refs/References.bib}

%% Common Parts

\newcommand{\progname}{MPIR} % PUT YOUR PROGRAM NAME HERE
\newcommand{\authname}{Xunzhou (Joe) Ye} % AUTHOR NAMES

\newcommand{\matr}[1]{\mathbf{#1}}
\renewcommand{\vec}[1]{\mathbf{#1}}
\newcommand{\spann}[1]{\mathrm{span}\{#1\}}

\usepackage{hyperref}
    \hypersetup{colorlinks=true, linkcolor=blue, citecolor=blue, filecolor=blue,
                urlcolor=blue, unicode=false}
    \urlstyle{same}


%\usetheme{Iimenau}

\useoutertheme{split} %so the footline can be seen, without needing pgfpages

%\pgfpagesuselayout{resize to}[letterpaper,border shrink=5mm,landscape]  %if this is uncommented, the hyperref links do not work

\mode<presentation>{}

\input{../../def-beamer}

\newcommand{\topic}{\progname{} Implementations}

%Title page information for 1D04 lectures slides

% Define year specific parameters - used in title page and footer

\newcommand{\season}{Winter} %use to switch between Winter and Fall
\newcommand{\instructor}{Xunzhou (Joe) Ye} %use to switch instructor
\newcommand{\instructSmall}{Xunzhou}
\newcommand{\yr}{2025}
\newcommand{\courseCode}{CAS 741}
\newcommand{\courseTitle}{Development of Scientific Computing Software}

%\setbeamerfont{structure}{series=\bfseries}
%\usefonttheme[stillsansseriftext,stillsansserifmath]{serif}
\setbeamertemplate{navigation symbols}{}
\setbeamertemplate{itemize item}[ball]

\title{
  {\normalsize \bf
    \borange{\courseCode~(\courseTitle)\\ \season~\yr}}\\[2ex]
  {\Large \bf \topic}}

\author[Smith]{\instructor}

\institute{
  Faculty of Engineering,
  McMaster University}

\date{
\today
\bc
  \includegraphics[scale = 0.2, keepaspectratio]
  {../mcmaster-logo-full-color.jpg}
\ec
}

\renewcommand{\borange}[1] %orange is too hard to read
{
   \bred{#1}
}


\begin{document}

% Footline for  Slides

% Display title page and displays footers

\setbeamertemplate{footline}{} %so the title screen does not have a footline

%%%%%%%%%%%%%%%%%%%%%%%%%%%%%%%%%%%%%%%%%%%%%%%%%%%%%%%%%%%%

\begin{frame}
\titlepage
\end{frame}

%%%%%%%%%%%%%%%%%%%%%%%%%%%%%%%%%%%%%%%%%%%%%%%%%%%%%%

\setbeamertemplate{footline}{
\begin{beamercolorbox}{sectioninhead/foot}
\hspace{1ex}\bblue{\hrulefill}\hspace{1ex}

\vspace{1ex}
\hspace{1ex}
{\tiny \instructSmall \hfill
\courseCode~\season~\yr:~\topic \hfill
\insertframenumber/\inserttotalframenumber~~}
%\insertframenumber/\ref{lastframe}}
%\hfill {\small \insertframenumber} \hspace{10ex}
%{\small $$\insertframenumber$$}
\vspace{1ex}
\end{beamercolorbox}}

%%%%%%%%%%%%%%%%%%%%%%%%%%%%%%%%%%%%%%%%%%%%%%%%%%%%%%


%%%%%%%%%%%%%%%%%%%%%%%%%%%%%%%%%%%%%%%%%%%%%%%%%%%%%%

\begin{frame}
\frametitle{General Information}

\begin{itemize}
\item \progname{} is a sparse linear solver designed to solve large, sparse real
  matrices efficiently.
\item It uses the General Minimal Residual (GMRES) method for internal matrix
  solves and iterative refinement techniques to improve both speed and accuracy.
\item Intended for use in computational science, engineering, and numerical
  analysis applications.
\item As a complete library suite, the software also includes example programs to
  demonstrate the solver interfaces and practical use cases of the solver.
\end{itemize}

\end{frame}

%%%%%%%%%%%%%%%%%%%%%%%%%%%%%%%%%%%%%%%%%%%%%%%%%%%%%%%%%%%%%%%%%%%%%%%%%%%%%

\begin{frame}
\frametitle{Inputs}

\begin{table}[hp]
  \centering
  \label{tab:inputs}
  \begin{tabularx}{1.0\linewidth}{rX}
    \toprule
    \textbf{Variable}  & \textbf{Description} \\
    \midrule
    \(\matr{A}\) & \(n \times n\) matrix \\
    \(\vec{b}\)        & \(n\)-vector \\
    \(\epsilon\)        & a solution is found if the norm of the residual is less than \(\epsilon\) \\
    \(n_\mathrm{iter}\) & the maximum number of iterations to perform \\
    \(u_f\)       & factorization precision \\
    \(u_w\)       & working precision \\
    \(u_r\)       & precision in which the residuals are computed \\
    \bottomrule
  \end{tabularx}
\end{table}

\end{frame}

%%%%%%%%%%%%%%%%%%%%%%%%%%%%%%%%%%%%%%%%%%%%%%%%%%%%%%%%%%%%%%%%%%%%%%%%%%%%%

\begin{frame}
\frametitle{Assumptions and Constraints}

\begin{enumerate}[{A}1]
\item Matrix \(\matr{A}\) is symmetric quasi-definite.
\item \(\matr{A}\) is stored in \cite{noauthor_compressed_nodate} format.
\item Only the upper triangular part of \(\matr{A}\) is stored.
\item The precisions follow the order \(u_f \leq u_w \leq u_r\), with \(u_r\) being the
  highest precision.
\end{enumerate}

\begin{enumerate}[{C}1]
\item Use preconditioned GMRES for solving the error correction vector.
\end{enumerate}

\end{frame}

%%%%%%%%%%%%%%%%%%%%%%%%%%%%%%%%%%%%%%%%%%%%%%%%%%%%%%%%%%%%%%%%%%%%%%%%%%%%%

\begin{frame}
\frametitle{Core Algorithms}

  \begin{algorithm}[H]
    \caption{Iterative refinement}
    \begin{algorithmic}[1]
      \For{\(m \gets 1, 2, \dots\), the \(m\)th iteration}
      \State \(\vec{r}_m \gets \vec{b} - \matr{A}\vec{x}_m\) \Comment{Compute the residuals}\label{algo:ir:residual}
      \State Solve \(\matr{A}\vec{d}_m = \vec{r}_m\) for \(\vec{d}_m\) \Comment{Compute the correction}\label{algo:ir:solve}
      \State \(\vec{x}_{m + 1} = \vec{x}_m + \vec{d}_m\) \Comment{Add the correction}
      \EndFor
    \end{algorithmic}
  \end{algorithm}
\end{frame}

%%%%%%%%%%%%%%%%%%%%%%%%%%%%%%%%%%%%%%%%%%%%%%%%%%%%%%%%%%%%%%%%%%%%%%%%%%%%%

\begin{frame}
\frametitle{Core Algorithms}

  \begin{algorithm}[H]
    \caption{GMRES-IR with \(\matr{L}\matr{D}\matr{L}\transpose\) factorization in MP}
    \begin{algorithmic}[1]
      \State Perform \(\matr{L}\matr{D}\matr{L}\transpose\) factorization of \(\matr{A}\) \Comment{at \(u_f\)}
      \State Solve \(\matr{L}\matr{D}\matr{L}\transpose \vec{x}_0 = \vec{b}\) \Comment{at \(u_f\)}
      \For{\(i \gets 0, 1, \dots, n_\mathrm{iter}\) and \(\norm{r_i}_2 \geq \epsilon\)}
      \State \(r_i \gets \vec{b} - \matr{A}\vec{x}_i\) \Comment{at \(u_r\)}
      \State GMRES Solve \((\matr{L}\matr{D}\matr{L}\transpose)^{-1}\matr{A}\vec{d}_i = (\matr{L}\matr{D}\matr{L}\transpose)^{-1}\vec{r_i}\) \Comment{at \(u_w\)}
      \State \(\vec{x}_{i + 1} = \vec{x}_i + \vec{d}_i\) \Comment{at \(u_r\)}
      \EndFor
    \end{algorithmic}
  \end{algorithm}
\end{frame}

%%%%%%%%%%%%%%%%%%%%%%%%%%%%%%%%%%%%%%%%%%%%%%%%%%%%%%%%%%%%%%%%%%%%%%%%%%%%%

\begin{frame}
\frametitle{Floating Point Precisions}

\begin{table}[H]
  \centering
  \begin{tabularx}{\linewidth}{lcccl} \toprule
    \textbf{Arithmetic} & \textbf{Sym.} & \multicolumn{2}{c}{\textbf{Bits}} & \textbf{Type}                                 \\ \cline{3-4}
                  &         & \textbf{Sig.}            & \textbf{Exp.} &                               \\ \midrule
    bfloat16      & B       & 8                  & 8       & \texttt{std::bfloat16\_t}           \\
    fp16          & H       & 11                 & 5       & \texttt{std::float16\_t}            \\
    fp32          & S       & 24                 & 8       & \texttt{std::float32\_t}, \texttt{float}  \\
    fp64          & D       & 53                 & 11      & \texttt{std::float64\_t}, \texttt{double} \\
    fp128         & Q       & 113                & 15      & \texttt{std::float128\_t}           \\ \bottomrule
  \end{tabularx}
\end{table}
\end{frame}

%%%%%%%%%%%%%%%%%%%%%%%%%%%%%%%%%%%%%%%%%%%%%%%%%%%%%%%%%%%%%%%%%%%%%%%%%%%%%

\begin{frame}
\frametitle{Demo}

\begin{itemize}
\item C++ templates: a bit of metaprogramming to ensure \(u_f \leq u_w \leq u_r\).
\item CMake/CTest: portability, project dependencies, unit test driver, CI/CD.
\item clang-format: also with CI/CD.
\end{itemize}

\end{frame}

%%%%%%%%%%%%%%%%%%%%%%%%%%%%%%%%%%%%%%%%%%%%%%%%%%%%%%%%%%%%%%%%%%%%%%%%%%%%%

\begin{frame}
\frametitle{Future Work}

\begin{enumerate}
\item Performance testing
\item Hardly converges in low precisions
\end{enumerate}

\end{frame}

%%%%%%%%%%%%%%%%%%%%%%%%%%%%%%%%%%%%%%%%%%%%%%%%%%%%%%%%%%%%%%%%%%%%%%%%%%%%%

\begin{frame}
\frametitle{References}

\printbibliography[heading=none]{}

\end{frame}

%%%%%%%%%%%%%%%%%%%%%%%%%%%%%%%%%%%%%%%%%%%%%%%%%%%%%%%%%%%%%%%%%%%%%%%%%%%%%

\begin{frame}
\frametitle{Questions}

\end{frame}

%%%%%%%%%%%%%%%%%%%%%%%%%%%%%%%%%%%%%%%%%%%%%%%%%%%%%%%%%%%%%%%%%%%%%%%%%%%%%

\end{document}
