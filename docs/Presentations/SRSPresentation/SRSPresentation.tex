%\documentclass[handout]{beamer}
\documentclass[t,12pt,numbers,fleqn]{beamer}
%\documentclass[ignorenonframetext]{beamer}

\newif\ifquestions
%\questionstrue
\questionsfalse

\usepackage{pgfpages}
\usepackage{hyperref}
\hypersetup{colorlinks=true,
    linkcolor=blue,
    citecolor=blue,
    filecolor=blue,
    urlcolor=blue,
    unicode=false}
\urlstyle{same}

\usepackage{booktabs}

\usepackage{caption}
\usepackage{subcaption}
\captionsetup[figure]{font=scriptsize,labelfont=scriptsize}

\usepackage[version=4]{mhchem}
\usepackage{siunitx}

%% Common Parts

\newcommand{\progname}{MPIR} % PUT YOUR PROGRAM NAME HERE
\newcommand{\authname}{Xunzhou (Joe) Ye} % AUTHOR NAMES

\newcommand{\matr}[1]{\mathbf{#1}}
\renewcommand{\vec}[1]{\mathbf{#1}}
\newcommand{\spann}[1]{\mathrm{span}\{#1\}}

\usepackage{hyperref}
    \hypersetup{colorlinks=true, linkcolor=blue, citecolor=blue, filecolor=blue,
                urlcolor=blue, unicode=false}
    \urlstyle{same}


%\usetheme{Iimenau}

\useoutertheme{split} %so the footline can be seen, without needing pgfpages

%\pgfpagesuselayout{resize to}[letterpaper,border shrink=5mm,landscape]  %if this is uncommented, the hyperref links do not work

\mode<presentation>{}

\input{../def-beamer}

\newcommand{\topic}{Mixed-Precision Iterative Solver}

%Title page information for 1D04 lectures slides

% Define year specific parameters - used in title page and footer

\newcommand{\season}{Winter} %use to switch between Winter and Fall
\newcommand{\instructor}{Xunzhou (Joe) Ye} %use to switch instructor
\newcommand{\instructSmall}{Xunzhou}
\newcommand{\yr}{2025}
\newcommand{\courseCode}{CAS 741}
\newcommand{\courseTitle}{Development of Scientific Computing Software}

%\setbeamerfont{structure}{series=\bfseries}
%\usefonttheme[stillsansseriftext,stillsansserifmath]{serif}
\setbeamertemplate{navigation symbols}{}
\setbeamertemplate{itemize item}[ball]

\title{
  {\normalsize \bf
    \borange{\courseCode~(\courseTitle)\\ \season~\yr}}\\[2ex]
  {\Large \bf \topic}}

\author[Smith]{\instructor}

\institute{
  Faculty of Engineering,
  McMaster University}

\date{
\today
\bc
  \includegraphics[scale = 0.2, keepaspectratio]
  {../mcmaster-logo-full-color.jpg}
\ec
}

\renewcommand{\borange}[1] %orange is too hard to read
{
   \bred{#1}
}


\begin{document}

% Footline for  Slides

% Display title page and displays footers

\setbeamertemplate{footline}{} %so the title screen does not have a footline

%%%%%%%%%%%%%%%%%%%%%%%%%%%%%%%%%%%%%%%%%%%%%%%%%%%%%%%%%%%%

\begin{frame}
\titlepage
\end{frame}

%%%%%%%%%%%%%%%%%%%%%%%%%%%%%%%%%%%%%%%%%%%%%%%%%%%%%%

\setbeamertemplate{footline}{
\begin{beamercolorbox}{sectioninhead/foot}
\hspace{1ex}\bblue{\hrulefill}\hspace{1ex}

\vspace{1ex}
\hspace{1ex}
{\tiny \instructSmall \hfill
\courseCode~\season~\yr:~\topic \hfill
\insertframenumber/\inserttotalframenumber~~}
%\insertframenumber/\ref{lastframe}}
%\hfill {\small \insertframenumber} \hspace{10ex}
%{\small $$\insertframenumber$$}
\vspace{1ex}
\end{beamercolorbox}}

%%%%%%%%%%%%%%%%%%%%%%%%%%%%%%%%%%%%%%%%%%%%%%%%%%%%%%


%%%%%%%%%%%%%%%%%%%%%%%%%%%%%%%%%%%%%%%%%%%%%%%%%%%%%%

\begin{frame}
\frametitle{Linear Solver: the Good Old \(\matr{A}\vec{x} = \vec{b}\) Problem}

\begin{itemize}
\item Direct method: Gaussian elimination
\end{itemize}

\[
  \begin{aligned}
    \matr{A}\vec{x} = &
    \left[
    \begin{array}{ccc}
      1 & -1 & 3 \\
      1 &  1 & 0 \\
      3 & -2 & 1
    \end{array}
    \right]
    \begin{bmatrix}
      x_1 \\
      x_2 \\
      x_3
    \end{bmatrix}
    =
    \begin{bmatrix}
      11 \\
      3 \\
      3
    \end{bmatrix}
    = \vec{b}
    \\
    \matr{A}|\vec{b} = &
    \left[
    \begin{array}{ccc|c}
      1 & -1 & 3 & 11 \\
      1 &  1 & 0 &  3 \\
      3 & -2 & 1 &  3
    \end{array}
    \right]
    \begin{array}{cc}
      \times 1 & \times 3 \\
      \downarrow   & \\
               & \downarrow
    \end{array}
    \\
    \matr{A}|\vec{b} \leftarrow &
    \left[
    \begin{array}{ccc|c}
      1 & -1 & 3 &  11 \\
      0 &  2 & -3 &  -8 \\
      0 &  1 & -8 & -30
    \end{array}
    \right]
  \end{aligned}
\]

\end{frame}

%%%%%%%%%%%%%%%%%%%%%%%%%%%%%%%%%%%%%%%%%%%%%%%%%%%%%%%%%%%%%%%%%%%%%%%%%%%%%

\begin{frame}
\frametitle{Think Big, Think Sparse}

\begin{itemize}
\item In structural simulations (e.g., finite element analysis), structures are
  modeled based on real-world physics, where forces and constraints are often
  localized.
\item Each element or node in a structure is typically connected only to a few
  nearby elements or nodes.
\item System of equations in structural simulations is assembled from local
  contributions.
\item In contrast, in neural networks, dense weight matrices are used in fully
  connected layers, particularly in large networks like transformers or dense
  layers after convolutional layers.
\end{itemize}

\end{frame}

%%%%%%%%%%%%%%%%%%%%%%%%%%%%%%%%%%%%%%%%%%%%%%%%%%%%%%%%%%%%%%%%%%%%%%%%%%%%%

\begin{frame}
\frametitle{Sparse Matrix Example: Boeing/ct20stif}

\begin{itemize}
\item \(\num{52329} \times \num{52329}\)
\item \num{2600295} non-zeros
\item \(\approx \SI{33}{\mega\byte}\) in file size
\end{itemize}

\begin{figure}[hh]
  \centering
  \begin{subfigure}[b]{0.5\textwidth}
    \centering
    \includegraphics[width=\textwidth,trim={2cm 1.5cm 2cm 1cm},clip]{figures/ct20stif}
  \end{subfigure}
  \begin{subfigure}[b]{0.4\textwidth}
    \centering
    \includegraphics[width=\textwidth]{figures/ct20stif_graph}
  \end{subfigure}
  \caption*{Boeing/ct20stif: CT20 Engine Block -- Stiffness
    matrix}
  \label{fig:ct20}
\end{figure}

\end{frame}

%%%%%%%%%%%%%%%%%%%%%%%%%%%%%%%%%%%%%%%%%%%%%%%%%%%%%%%%%%%%%%%%%%%%%%%%%%%%%

\begin{frame}
\frametitle{Sparse Matrix Example: Janna/Serena}

\begin{itemize}
\item \(\num{1391349} \times \num{1391349}\)
\item \num{64131971} non-zeros
\item \(\approx \SI{847}{\mega\byte}\) in file size
\end{itemize}

\begin{figure}[hh]
  \centering
  \begin{subfigure}[b]{0.5\textwidth}
    \centering
    \includegraphics[width=\textwidth,trim={2cm 1.5cm 2cm 1cm},clip]{figures/Serena}
  \end{subfigure}
  \begin{subfigure}[b]{0.37\textwidth}
    \centering
    \includegraphics[width=\textwidth]{figures/Serena_graph}
  \end{subfigure}
  \caption*{Janna/Serena: gas resevoir simulation for \ce{CO2}
    sequestration}
  \label{fig:Serena}
\end{figure}

\end{frame}

%%%%%%%%%%%%%%%%%%%%%%%%%%%%%%%%%%%%%%%%%%%%%%%%%%%%%%%%%%%%%%%%%%%%%%%%%%%%%

\begin{frame}
\frametitle{The Problem with Gaussian Elimination}

\begin{itemize}
\item Store the whole matrix
\item Memory bounded
\item Sparsity preservation
\end{itemize}

\begin{figure}[h]
  \centering
  \begin{subfigure}[b]{0.5\textwidth}
    \centering
    \includegraphics[width=\textwidth,trim={2cm 1.5cm 2cm 1cm},clip]{figures/Serena}
  \end{subfigure}
  \begin{subfigure}[b]{0.4\textwidth}
    \centering
    \includegraphics[width=\textwidth]{figures/chicken}
  \end{subfigure}
  \caption*{Janna/Serena: gas resevoir simulation for \ce{CO2}
    sequestration}
  \label{fig:Serena}
\end{figure}

\end{frame}

%%%%%%%%%%%%%%%%%%%%%%%%%%%%%%%%%%%%%%%%%%%%%%%%%%%%%%%%%%%%%%%%%%%%%%%%%%%%%

\end{document}
