%\documentclass[handout]{beamer}
\documentclass[t,12pt,numbers,fleqn]{beamer}
%\documentclass[ignorenonframetext]{beamer}

\newif\ifquestions
%\questionstrue
\questionsfalse

\usepackage{pgfpages}
\usepackage{hyperref}
\hypersetup{colorlinks=true,
    linkcolor=blue,
    citecolor=blue,
    filecolor=blue,
    urlcolor=blue,
    unicode=false}
\urlstyle{same}

\usepackage{booktabs}

\usepackage{caption}
\usepackage{subcaption}
\captionsetup[figure]{font=scriptsize,labelfont=scriptsize}

\usepackage[version=4]{mhchem}
\usepackage[print-unity-mantissa=false]{siunitx}
\usepackage{tabularx}


\usepackage[backend=biber,style=authoryear]{biblatex}
\addbibresource{../../../refs/References.bib}

%% Common Parts

\newcommand{\progname}{MPIR} % PUT YOUR PROGRAM NAME HERE
\newcommand{\authname}{Xunzhou (Joe) Ye} % AUTHOR NAMES

\newcommand{\matr}[1]{\mathbf{#1}}
\renewcommand{\vec}[1]{\mathbf{#1}}
\newcommand{\spann}[1]{\mathrm{span}\{#1\}}

\usepackage{hyperref}
    \hypersetup{colorlinks=true, linkcolor=blue, citecolor=blue, filecolor=blue,
                urlcolor=blue, unicode=false}
    \urlstyle{same}


%\usetheme{Iimenau}

\useoutertheme{split} %so the footline can be seen, without needing pgfpages

%\pgfpagesuselayout{resize to}[letterpaper,border shrink=5mm,landscape]  %if this is uncommented, the hyperref links do not work

\mode<presentation>{}

\input{../../def-beamer}

\newcommand{\topic}{\progname{} Verification and Validation Plan}

%Title page information for 1D04 lectures slides

% Define year specific parameters - used in title page and footer

\newcommand{\season}{Winter} %use to switch between Winter and Fall
\newcommand{\instructor}{Xunzhou (Joe) Ye} %use to switch instructor
\newcommand{\instructSmall}{Xunzhou}
\newcommand{\yr}{2025}
\newcommand{\courseCode}{CAS 741}
\newcommand{\courseTitle}{Development of Scientific Computing Software}

%\setbeamerfont{structure}{series=\bfseries}
%\usefonttheme[stillsansseriftext,stillsansserifmath]{serif}
\setbeamertemplate{navigation symbols}{}
\setbeamertemplate{itemize item}[ball]

\title{
  {\normalsize \bf
    \borange{\courseCode~(\courseTitle)\\ \season~\yr}}\\[2ex]
  {\Large \bf \topic}}

\author[Smith]{\instructor}

\institute{
  Faculty of Engineering,
  McMaster University}

\date{
\today
\bc
  \includegraphics[scale = 0.2, keepaspectratio]
  {../mcmaster-logo-full-color.jpg}
\ec
}

\renewcommand{\borange}[1] %orange is too hard to read
{
   \bred{#1}
}


\begin{document}

% Footline for  Slides

% Display title page and displays footers

\setbeamertemplate{footline}{} %so the title screen does not have a footline

%%%%%%%%%%%%%%%%%%%%%%%%%%%%%%%%%%%%%%%%%%%%%%%%%%%%%%%%%%%%

\begin{frame}
\titlepage
\end{frame}

%%%%%%%%%%%%%%%%%%%%%%%%%%%%%%%%%%%%%%%%%%%%%%%%%%%%%%

\setbeamertemplate{footline}{
\begin{beamercolorbox}{sectioninhead/foot}
\hspace{1ex}\bblue{\hrulefill}\hspace{1ex}

\vspace{1ex}
\hspace{1ex}
{\tiny \instructSmall \hfill
\courseCode~\season~\yr:~\topic \hfill
\insertframenumber/\inserttotalframenumber~~}
%\insertframenumber/\ref{lastframe}}
%\hfill {\small \insertframenumber} \hspace{10ex}
%{\small $$\insertframenumber$$}
\vspace{1ex}
\end{beamercolorbox}}

%%%%%%%%%%%%%%%%%%%%%%%%%%%%%%%%%%%%%%%%%%%%%%%%%%%%%%


%%%%%%%%%%%%%%%%%%%%%%%%%%%%%%%%%%%%%%%%%%%%%%%%%%%%%%

\begin{frame}
\frametitle{General Information}

\begin{itemize}
\item \progname{} is a sparse linear solver designed to solve large, sparse real
  matrices efficiently.
\item It uses the General Minimal Residual (GMRES) method for internal matrix
  solves and iterative refinement techniques to improve both speed and accuracy.
\item Intended for use in computational science, engineering, and numerical
  analysis applications.
\item As a complete library suite, the software also includes example programs to
  demonstrate the solver interfaces and practical use cases of the solver.
\end{itemize}

\end{frame}

%%%%%%%%%%%%%%%%%%%%%%%%%%%%%%%%%%%%%%%%%%%%%%%%%%%%%%%%%%%%%%%%%%%%%%%%%%%%%

\begin{frame}
\frametitle{Inputs (Partial)}

\begin{table}[hp]
  \centering
  \label{tab:inputs}
  \begin{tabularx}{1.0\linewidth}{rX}
    \toprule
    \textbf{Variable}  & \textbf{Description} \\
    \midrule
    \(\matr{A}\) & \(n \times n\) matrix \\
    \(\vec{b}\)        & \(n\)-vector \\
    \(u_f\)       & factorization precision \\
    \(u_w\)       & working precision \\
    \(u_r\)       & precision in which the residuals are computed \\
    \bottomrule
  \end{tabularx}
\end{table}

\end{frame}

%%%%%%%%%%%%%%%%%%%%%%%%%%%%%%%%%%%%%%%%%%%%%%%%%%%%%%%%%%%%%%%%%%%%%%%%%%%%%

\begin{frame}
\frametitle{VnV Objectives}

\begin{description}
\item[Primary] Ensure the correctness, accuracy, and efficiency of \progname{} in
  solving sparse linear systems.
\item[Secondary] Verify usability and maintainability of the software for
  integration with other numerical libraries.
\item[Out of Scope] Usability testing for non-expert users is not prioritized.
  External libraries used for matrix factorization, reading and writing sparse
  matrices in Matrix Market Exchange Format (\cite{noauthor_matrix_2013}), unit
  testing, and performance benchmarking are assumed to be correct.
\end{description}

\end{frame}

%%%%%%%%%%%%%%%%%%%%%%%%%%%%%%%%%%%%%%%%%%%%%%%%%%%%%%%%%%%%%%%%%%%%%%%%%%%%%

\begin{frame}
\frametitle{VnV Team}

\vspace{10pt}
\begin{tabularx}{\linewidth}{lX}
  \toprule
  \textbf{Team Member} & \textbf{Role}                           \\
  \midrule
  Joe Ye         & Lead developer and tester         \\
  Maris Chen     & ``Domain expert'', provides feedbacks on documents per course
                   guidelines and document templates \\
  Dr. Nedialkov  & Primary stakeholder, oversees project direction and validates
                   all documents                     \\
  \bottomrule
\end{tabularx}

\end{frame}

%%%%%%%%%%%%%%%%%%%%%%%%%%%%%%%%%%%%%%%%%%%%%%%%%%%%%%%%%%%%%%%%%%%%%%%%%%%%%

\begin{frame}
\frametitle{Functional System Testing: Correctness and Accuracy}

\begin{itemize}
\item Define acceptance criteria in terms of machine epsilon, the smallest
  difference in value that the computer can tell apart at a given precision.
\item The condition number \(\cond{\matr{A}}\) is an important factor when
  choosing inputs.
\end{itemize}

\vspace{10pt}
\begin{tabularx}{\textwidth}{Xl}
  \toprule
  \textbf{Floating-Point Format} & \textbf{Machine Epsilon (\(\epsilon_\mathrm{mach}\))} \\
  \midrule
  bfloat16                 & \num{3.91e-3}                     \\
  fp16 (IEEE 754 Half)     & \num{4.88e-4}                     \\
  fp32 (IEEE 754 Single)   & \num{1.19e-7}                     \\
  fp64 (IEEE 754 Double)   & \num{2.22e-16}                    \\
  fp128 (IEEE 754 Quad)    & \num{1.93e-34}                    \\
  \bottomrule
\end{tabularx}

\end{frame}

%%%%%%%%%%%%%%%%%%%%%%%%%%%%%%%%%%%%%%%%%%%%%%%%%%%%%%%%%%%%%%%%%%%%%%%%%%%%%

\begin{frame}
\frametitle{The Known Solution Vector Approach}

Test against manufactured solutions:
\begin{enumerate}
\item \(\vec{x}_\mathrm{ref} \gets \text{random vector}\)
\item \(\vec{b} \gets \matr{A} \vec{x}_\mathrm{ref} \)
\item Solve \(\matr{A}\vec{x} = \vec{b}\)
\item \(e \gets \displaystyle \frac{\norm{\vec{x} - \vec{x}_\mathrm{ref}}_2}{\norm{\vec{x}_\mathrm{ref}}_2}\)
\end{enumerate}

Caveats:
\begin{itemize}
\item Large error does not necessarily mean the solver is incorrect. If
  \(\cond{\matr{A}}\) is large, the error is likely to be large as well.
\item For well-conditioned matrices, e.g. \(\cond{\matr{A}} < \num{1e4}\), the
  relative error is expected to be \(e \approx \num{1e-12}\) in double precision
  (\(\epsilon_\mathrm{mach} \approx \num{2.2e-16}\))
\end{itemize}

\end{frame}

%%%%%%%%%%%%%%%%%%%%%%%%%%%%%%%%%%%%%%%%%%%%%%%%%%%%%%%%%%%%%%%%%%%%%%%%%%%%%

\begin{frame}
\frametitle{The Trusted Solver Comparison Approach}

\begin{itemize}
\item MATLAB\textsuperscript{\textregistered{}} \texttt{vpa}
\item Other readily available sparse solvers
\end{itemize}

The relative error, \(e = \displaystyle \frac{\norm{\vec{x} -
    \vec{x}_\mathrm{ref}}_2}{\norm{\vec{x}_\mathrm{ref}}_2}\) should be within a
reasonable multiple of \(\epsilon_\mathrm{mach}\) across a set of test problems.

\end{frame}

%%%%%%%%%%%%%%%%%%%%%%%%%%%%%%%%%%%%%%%%%%%%%%%%%%%%%%%%%%%%%%%%%%%%%%%%%%%%%

\begin{frame}
\frametitle{Functional System Testing: the Plan}

\begin{enumerate}
\item Start with numerically stable inputs (well conditioned \(\matr{A}\)) without
  mixed-precisions. Verify accuracy with both the ``known solution vector'' and
  ``trusted solver comparison'' approach.
\item Hypothesize that the solver is able to solve systems with condition number up
  to \textit{COND\_MAX}. Test robustness by inputting ill-conditioned matrices
  with \(\cond{\matr{A}}\) at different magnitudes. (Non-functional?)
\item Hypothesize that the solver is able to maintain accuracy when mixed-precision
  technique is applied. Test accuracy by inputting different combinations of
  \(u_f, u_w, u_r\).
\end{enumerate}

\end{frame}

%%%%%%%%%%%%%%%%%%%%%%%%%%%%%%%%%%%%%%%%%%%%%%%%%%%%%%%%%%%%%%%%%%%%%%%%%%%%%

\begin{frame}
\frametitle{Functional System Testing: an Example}

\begin{enumerate}

\item[test-id1]

Control: Automatic

Initial State: matrix \(\matr{A}\) is read from file and stored in memory.
Expected exact solution \(\vec{x}_\mathrm{ref}\) is prepared.

Input: matrix \(\matr{A}\) of size \(\num{10000} \times \num{10000}\) with
\(\cond{\matr{A}} \approx \num{1e2}\), \(\vec{b}\) of size \num{10000}, \(u_f =
u_w = u_r = \texttt{double}\)

Output: \(\vec{x}\) of size \num{10000} such that \(e = \displaystyle \frac{\norm{\vec{x} -
    \vec{x}_\mathrm{ref}}_2}{\norm{\vec{x}_\mathrm{ref}}_2} < \num{1e-12}\)

Test Case Derivation: \(\vec{x}_\mathrm{ref}\) is randomly generated. \(\vec{b} = \matr{A}\vec{x}_\mathrm{ref}\)

How test will be performed: Automatic

\end{enumerate}
\end{frame}

%%%%%%%%%%%%%%%%%%%%%%%%%%%%%%%%%%%%%%%%%%%%%%%%%%%%%%%%%%%%%%%%%%%%%%%%%%%%%

\begin{frame}
\frametitle{Non-Functional System Testing: Performance}

Choice of performance metric?

\begin{itemize}
\item Convergence rate?
\item Computation time/Wall-clock time
\end{itemize}

\end{frame}

%%%%%%%%%%%%%%%%%%%%%%%%%%%%%%%%%%%%%%%%%%%%%%%%%%%%%%%%%%%%%%%%%%%%%%%%%%%%%

\begin{frame}
\frametitle{Potential Problems with Measuring Runtime}

\begin{itemize}
\item CPU clock frequency, temperature; Caching
\item OS scheduling, interrupts, context switching
\item Hardware optimizations specifically for certain arithmetic/precision; CPU vs.
  GPU
\item Inconsistency between runs, hard to reproduce results
\end{itemize}

\end{frame}

%%%%%%%%%%%%%%%%%%%%%%%%%%%%%%%%%%%%%%%%%%%%%%%%%%%%%%%%%%%%%%%%%%%%%%%%%%%%%

\begin{frame}
\frametitle{Established Tools for Benchmarking}

\cite{noauthor_googlebenchmark_2025}
\begin{itemize}
\item Repeats benchmark cases until a statistically stable result is obtained
\item Automatic CPU frequency scaling adjustments
\end{itemize}

\end{frame}

%%%%%%%%%%%%%%%%%%%%%%%%%%%%%%%%%%%%%%%%%%%%%%%%%%%%%%%%%%%%%%%%%%%%%%%%%%%%%

\begin{frame}
\frametitle{Non-Functional System Testing: the Plan}

\begin{enumerate}
\item Hypothesize that the use of mixed-precision technique brings consistent
  improvement in runtime. Given well-conditioned inputs, benchmark runtime by
  varying \(u_f, u_w, u_r\).
\end{enumerate}

\end{frame}

%%%%%%%%%%%%%%%%%%%%%%%%%%%%%%%%%%%%%%%%%%%%%%%%%%%%%%%%%%%%%%%%%%%%%%%%%%%%%

\begin{frame}
\frametitle{References}

\printbibliography[heading=none]{}

\end{frame}

%%%%%%%%%%%%%%%%%%%%%%%%%%%%%%%%%%%%%%%%%%%%%%%%%%%%%%%%%%%%%%%%%%%%%%%%%%%%%

\begin{frame}
\frametitle{Questions}

\end{frame}

%%%%%%%%%%%%%%%%%%%%%%%%%%%%%%%%%%%%%%%%%%%%%%%%%%%%%%%%%%%%%%%%%%%%%%%%%%%%%

\end{document}
