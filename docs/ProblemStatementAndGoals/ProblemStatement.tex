\documentclass{article}

\usepackage{tabularx}
\usepackage{booktabs}

\title{Problem Statement and Goals\\\progname}

\author{\authname}

\date{}

%% Comments

\usepackage{color}

\newif\ifcomments\commentstrue %displays comments
%\newif\ifcomments\commentsfalse %so that comments do not display

\ifcomments
\newcommand{\authornote}[3]{\textcolor{#1}{[#3 ---#2]}}
\newcommand{\todo}[1]{\textcolor{red}{[TODO: #1]}}
\else
\newcommand{\authornote}[3]{}
\newcommand{\todo}[1]{}
\fi

\newcommand{\wss}[1]{\authornote{blue}{SS}{#1}} 
\newcommand{\plt}[1]{\authornote{magenta}{TPLT}{#1}} %For explanation of the template
\newcommand{\an}[1]{\authornote{cyan}{Author}{#1}}

%% Common Parts

\newcommand{\progname}{MPIR} % PUT YOUR PROGRAM NAME HERE
\newcommand{\authname}{Xunzhou (Joe) Ye} % AUTHOR NAMES

\newcommand{\matr}[1]{\mathbf{#1}}
\renewcommand{\vec}[1]{\mathbf{#1}}

\usepackage{hyperref}
    \hypersetup{colorlinks=true, linkcolor=blue, citecolor=blue, filecolor=blue,
                urlcolor=blue, unicode=false}
    \urlstyle{same}


\begin{document}

\maketitle

\begin{table}[hp]
  \caption{Revision History} \label{TblRevisionHistory}
  \begin{tabularx}{\textwidth}{llX}
    \toprule
    \textbf{Date} & \textbf{Developer(s)} & \textbf{Change}\\
    \midrule
    \date{16 January 2025} & Xunzhou & Initial draft \\
    \bottomrule
  \end{tabularx}
\end{table}

\section{Problem Statement}

\wss{You should check your problem statement with the
\href{https://github.com/smiths/capTemplate/blob/main/docs/Checklists/ProbState-Checklist.pdf}
{problem statement checklist}.}

\wss{You can change the section headings, as long as you include the required
information.}

\subsection{Problem}

In numerical computing, mixed precision \emph{iterative refinement (IR)} is a
technique to solve systems of linear equations efficiently while maintaining
high accuracy. It combines computations in lower precision (e.g., single or half
precision) for speed and resource efficiency, with higher precision (e.g.,
double precision) for key operations to ensure numerical stability and accuracy.
In sparse matrix solvers, where computational costs and memory access patterns
are critical, mixed precision IR balances speed and precision, making it ideal
for large-scale simulations, finite element methods, or optimization problems.
Lower precision calculations performed in the refinement process require less
memory bandwidth and storage, enabling better utilization of memory hierarchies
and allowing larger problems to fit into limited memory. These calculations also
run faster on modern hardware, particularly GPUs and tensor cores.

Dr. N. Nedialkov and his student Yiqi Huang have done some preliminary works in
developing a linear solver for quasi-definite matrices using mixed precision IR,
where the internal refinement steps utilizes the \emph{Generalized Minimal
  Residual (GMRES)} iterative method. The scope of this project is to refactor
and improve on the existing implementations.

\subsection{Inputs and Outputs}

\wss{Characterize the problem in terms of ``high level'' inputs and outputs.
Use abstraction so that you can avoid details.}

\paragraph{Inputs}

Like any general purposed numeric linear solver, this solver attempts to solve
\(x\) satisfying the equation \(\matr{A}x = b\) for some given matrix
\(\matr{A}\) and column vector \(b\) within certain tolerance \(\epsilon\). With the
added feature of employing a mixed precision technique, the user can customize
on the precision on which they wish to perform matrix factorizations, triangular
solves, and computing residuals. The list of inputs is summarized in table
\ref{tab:inputs}.

\begin{table}[hp]
  \centering
  \caption{List of inputs and parameters}
  \label{tab:inputs}
  \begin{tabularx}{1.0\linewidth}{rX}
    \toprule
    \textbf{Variable}  & \textbf{Description} \\
    \midrule
    \(\matr{A}\) & the linear system to be solved \\
    \(b\)        & one or many column vector(s) describing the problem \\
    \(\epsilon\)        & a solution is found if the residual is less than this number \\
    \(n_\mathrm{iter}\) & the maximum number of iterations to perform \\
    FACTPREC     & matrix factorization precision \\
    WORKPREC     & working precision used in triangular solves \\
    RESPREC      & precision in which the residuals are computed \\
    \bottomrule
  \end{tabularx}
\end{table}

\paragraph{outputs}

This solver outputs a numerical approximation of \(x\) to the problem
\(\matr{A}x = b\) along with some diagnostic information such as the resulting
residual of the solve, the number of iterations performed.

\subsection{Stakeholders}

\begin{itemize}
\item Supervisor of the project, Dr. N. Nedialkov.
\item As the library will be publicly released under an open source license, it is
  relevant to any individual who is interested in and can make use of
  high-performance sparse linear solvers for either academic or commercial
  purposes.
\end{itemize}

\subsection{Environment}

\wss{Hardware and software environment}

\begin{description}
\item[Software] Any general purpose operating system (OS) with compatible toolchain
  for building the library, optionally running a few example programs.
\item[Hardware] Computer with modern processors.
\end{description}

\section{Goals}

\begin{description}
\item[Basic linear solver function] Given some matrix \(\matr{A}\) and column
  vector \(b\), the solver should iteratively find \(x\) satisfying the equation
  \(\matr{A}x = b\) until the residual \(r = \matr{A}x - b\) is smaller than
  some tolerance \(\epsilon\) or the maximum number of iterations
  \(n_\mathrm{iter}\) is exhausted, whichever comes first.
\item[Mixed precision] Given some combinations of floating point precision
  configuration, the solver should perform internal steps such as matrix
  factorizations, triangular solves, and residual computation in these
  configured precisions.
\item[High performance and high efficiency] The solver should offer a quantifiable
  performance or resource utilization advantage over other competing sparse
  linear solvers.
\item[Ease of use] The library should offer a set of streamlined public application
  programming interfaces (APIs), such that when integrated into other software
  as a dependent library, the interfaces are self-contained, readable and easy
  to consume.
\item[Implementation improvement, educational] Based on existing implementations,
  the project should result in an out-of-box, production grade codebase with
  cross-platform support. As the codebase is developed in an academic setting,
  the codebase should capture domain knowledge in various forms including
  documentations, code comments, and academic reports. It should also
  demonstrate best practices to develop high quality software, including code
  modularity, traceability, maintainability, and so on.
\end{description}

\section{Stretch Goals}

\begin{description}
\item[Accuracy improvement] With the existing solver implementation
  as being the baseline, the refactored solver should produce more accurate
  results (residuals lower by at least 1 order of magnitude).
\item[Algorithm optimization] The solver should optimize existing algorithms such
  that given the same set of inputs, it produces results with the same accuracy
  using notably less time.
\end{description}

\section{Challenge Level and Extras}

\wss{State your expected challenge level (advanced, general or basic).  The
challenge can come through the required domain knowledge, the implementation or
something else.  Usually the greater the novelty of a project the greater its
challenge level.  You should include your rationale for the selected level.
Approval of the level will be part of the discussion with the instructor for
approving the project.  The challenge level, with the approval (or request) of
the instructor, can be modified over the course of the term.}

\wss{Teams may wish to include extras as either potential bonus grades, or to
make up for a less advanced challenge level.  Potential extras include usability
testing, code walkthroughs, user documentation, formal proof, GenderMag
personas, Design Thinking, etc.  Normally the maximum number of extras will be
two.  Approval of the extras will be part of the discussion with the instructor
for approving the project.  The extras, with the approval (or request) of the
instructor, can be modified over the course of the term.}

The challenge level of this project is expected to be general, leaning toward
the advanced level for the following reasons:
\begin{enumerate}
\item This is a continuation of a Master's level research project. The associated
  thesis is still working in progress.
\item The choice of algorithm and mathematical technique is predefined as part of
  the project. The mathematical knowledge required for understanding and
  implementing these algorithms (e.g. GMRES) is graduate level.
\item Working with multiple floating point precisions requires knowledge on
  computer architectures, low-level floating point representations in computer,
  as well as advance implementation techniques to support manipulations of
  precisions.
\item Testing, benchmarking, and evaluating the performance gain on applying these
  advanced algorithms and techniques are also very involved.
\end{enumerate}

% \newpage{}

% \section*{Appendix --- Reflection}

% \wss{Not required for CAS 741}

% The purpose of reflection questions is to give you a chance to assess your own
learning and that of your group as a whole, and to find ways to improve in the
future. Reflection is an important part of the learning process.  Reflection is
also an essential component of a successful software development process.  

Reflections are most interesting and useful when they're honest, even if the
stories they tell are imperfect. You will be marked based on your depth of
thought and analysis, and not based on the content of the reflections
themselves. Thus, for full marks we encourage you to answer openly and honestly
and to avoid simply writing ``what you think the evaluator wants to hear.''

Please answer the following questions.  Some questions can be answered on the
team level, but where appropriate, each team member should write their own
response:


% \begin{enumerate}
%     \item What went well while writing this deliverable?
%     \item What pain points did you experience during this deliverable, and how
%     did you resolve them?
%     \item How did you and your team adjust the scope of your goals to ensure
%     they are suitable for a Capstone project (not overly ambitious but also of
%     appropriate complexity for a senior design project)?
% \end{enumerate}

\end{document}
