\documentclass[12pt, titlepage]{article}

\usepackage{amsmath, mathtools}

\usepackage{amsfonts}
\usepackage{amssymb}
\usepackage{graphicx}
\usepackage{colortbl}
\usepackage{xr,xr-hyper}
\usepackage{hyperref}
\usepackage{longtable}
\usepackage{xfrac}
\usepackage{tabularx}
\usepackage{float}
\usepackage{siunitx}
\usepackage{booktabs}
\usepackage{multirow}
\usepackage[section]{placeins}
\usepackage{caption}
\usepackage{fullpage}
\usepackage{enumitem}
\usepackage{algpseudocode,algorithm}

\hypersetup{
  bookmarks=true,    % show bookmarks bar?
  colorlinks=true,   % false: boxed links; true: colored links
  linkcolor=red,     % color of internal links (change box color with linkbordercolor)
  citecolor=blue,    % color of links to bibliography
  filecolor=magenta, % color of file links
  urlcolor=cyan      % color of external links
}

\usepackage{array}

\externaldocument{../../SRS/SRS}
\newcommand{\iref}[1]{IM\ref{#1}}

%% Comments

\usepackage{color}

\newif\ifcomments\commentstrue %displays comments
%\newif\ifcomments\commentsfalse %so that comments do not display

\ifcomments
\newcommand{\authornote}[3]{\textcolor{#1}{[#3 ---#2]}}
\newcommand{\todo}[1]{\textcolor{red}{[TODO: #1]}}
\else
\newcommand{\authornote}[3]{}
\newcommand{\todo}[1]{}
\fi

\newcommand{\wss}[1]{\authornote{blue}{SS}{#1}} 
\newcommand{\plt}[1]{\authornote{magenta}{TPLT}{#1}} %For explanation of the template
\newcommand{\an}[1]{\authornote{cyan}{Author}{#1}}

%% Common Parts

\newcommand{\progname}{MPIR} % PUT YOUR PROGRAM NAME HERE
\newcommand{\authname}{Xunzhou (Joe) Ye} % AUTHOR NAMES

\newcommand{\matr}[1]{\mathbf{#1}}
\renewcommand{\vec}[1]{\mathbf{#1}}

\usepackage{hyperref}
    \hypersetup{colorlinks=true, linkcolor=blue, citecolor=blue, filecolor=blue,
                urlcolor=blue, unicode=false}
    \urlstyle{same}


\usepackage[backend=biber,style=authoryear]{biblatex}
\addbibresource{../../../refs/References.bib}

\begin{document}

\title{Module Interface Specification for \progname{}}

\author{\authname}

\date{\today}

\maketitle

\pagenumbering{roman}

\section{Revision History}

\begin{tabularx}{\textwidth}{p{3cm}p{2cm}X}
\toprule {\bf Date}  & {\bf Version} & {\bf Notes}                   \\
\midrule
\date{19 March 2025} & 1.0           & Initial draft                 \\
\date{16 April 2025} & 1.1           & Refine according to feedbacks \\
\bottomrule
\end{tabularx}

~\newpage

\section{Symbols, Abbreviations and Acronyms}

See SRS Documentation at \cite{SRS}

\newpage

\tableofcontents

\newpage

\pagenumbering{arabic}

\section{Introduction}

The following document details the Module Interface Specifications for
\progname{}. It is intended to solve a sparse linear system with an iterative
method in mixed-precisions.

Complementary documents include the System Requirement Specifications and Module
Guide. The full documentation and implementation can be found at \cite{SRS} and
\cite{MG}.

\section{Notation}

\wss{You should describe your notation.  You can use what is below as
  a starting point.}

The structure of the MIS for modules comes from \cite{HoffmanAndStrooper1995}, with
the addition that template modules have been adapted from \cite{GhezziEtAl2003}.
The mathematical notation comes from Chapter 3 of \cite{HoffmanAndStrooper1995}.
For instance, the symbol := is used for a multiple assignment statement and
conditional rules follow the form
$(c_1 \Rightarrow r_1 | c_2 \Rightarrow r_2 | ... | c_n \Rightarrow r_n )$.

The following table summarizes the primitive data types used by \progname.

\begin{center}
  \renewcommand{\arraystretch}{1.2}
  \noindent
  \begin{tabular}{l l p{7.5cm}}
    \toprule
    \textbf{Data Type}   & \textbf{Notation} & \textbf{Description}                                         \\
    \midrule
    character      & char        & a single symbol or digit                               \\
    integer        & $\mathbb{Z}$        & a number without a fractional component in (-$\infty$, $\infty$) \\
    natural number & $\mathbb{N}$        & a number without a fractional component in [1, $\infty$)    \\
    real           & $\mathbb{R}$        & any number in (-$\infty$, $\infty$)                              \\
    \bottomrule
  \end{tabular}
\end{center}

\noindent
The specification of \progname \ uses some derived data types: sequences, strings, and
tuples. Sequences are lists filled with elements of the same data type. Strings
are sequences of characters. Tuples contain a list of values, potentially of
different types. In addition, \progname \ uses functions, which
are defined by the data types of their inputs and outputs. Local functions are
described by giving their type signature followed by their specification.

Variables of matrices or vectors are in math bold face. For any matrix
\(\matr{A}\) or vector \(\vec{b}\), one with subscript \(\matr{A}_i\) or
\(\vec{b}_i\) always means ``the \(i\)th matrix/vector''. \(a_{i,j}\) or
\(b_{i}\) is used to reference ``the element at row \(i\) column \(j\) in matrix
\(\matr{A}\)'' or ``the \(i\)th element in vector \(\vec{b}\)''.

\section{Module Decomposition}

The following table is taken directly from the Module Guide document for this project.

\begin{table}[H]
  \centering
  \begin{tabular}{p{0.3\textwidth} p{0.6\textwidth}}
    \toprule
    \textbf{Level 1}                                            & \textbf{Level 2}                     \\
    \midrule
    {Hardware-Hiding Module}                              & --                             \\
    \midrule
    \multirow{4}{0.3\textwidth}{Behaviour-Hiding Module}  & Floating Point Concepts Module \\
                                                          & Matrix Operations Module       \\
                                                          & Factorization Module           \\
                                                          & Iterative Solver Module        \\
    \midrule
    \multirow{1}{0.3\textwidth}{Software Decision Module} &                                \\
    \bottomrule
  \end{tabular}
  \caption{Module Hierarchy}
  \label{TblMH}
\end{table}

\newpage

\section{MIS of Factorization Module} \label{M:factor}

\subsection{Module}

qdldl\footnote{This module was originally implemented in prior research work
  (\cite{shahrooz_derakhshan_using_2023}), and the corresponding source code is
  provided as part of this project. The specifications presented here describe
  the exported routines and their intended usage, without delving into the
  underlying mathematical models or state transitions.}

\subsection{Uses}

None

\subsection{Syntax}

\subsubsection{Exported Access Programs}

\begin{center}
  \begin{tabularx}{\linewidth}{%
    p{3cm}
    >{\raggedright\arraybackslash}X
    >{\raggedright\arraybackslash}l
    >{\raggedright\arraybackslash}l
    }
    \hline
    \textbf{Name}       & \textbf{In}                                           & \textbf{Out}                          & \textbf{Exceptions} \\
    \hline
    QDLDL\_etree  & \(\matr{A}: \mathbb{R}^{n \times n}\)                         & \(L_\mathrm{nz}: \mathbb{N}, \vec{E}: \mathbb{R}^n\)          & NOT\_UPPER    \\
    QDLDL\_factor & \(\matr{A}: \mathbb{R}^{n \times n}, L_\mathrm{nz}: \mathbb{N}, \vec{E}: \mathbb{R}^n, u_f\) & \(\matr{L}: \mathbb{R}^{n \times n}, \vec{d}: \mathbb{R}^n\) & FAC\_FAILED   \\
    QDLDL\_solve  & \(\matr{L}: \mathbb{R}^{n \times n}, \vec{d}: \mathbb{R}^n, \vec{b}: \mathbb{R}^n, u_w\)     & \(\vec{x}: \mathbb{R}^n\)                      & --            \\
    \hline
  \end{tabularx}
\end{center}

\subsection{Semantics}

\subsubsection{State Variables}

None

\subsubsection{Assumptions}

The (sparse) matrix used or returned by this module is stored in Compressed
Sparse Column (CSC) format (\cite{noauthor_compressed_nodate}).

\subsubsection{Access Routine Semantics}

\noindent QDLDL\_etree(\(\matr{A}\)):
\begin{itemize}
\item output: \(\vec{E} \coloneq\) elimination tree\footnote{An elimination tree is a directed
    tree that encodes the dependencies between columns of a sparse symmetric
    matrix during factorization. Each node in the tree corresponds to a column
    of the matrix, and the edges represent the flow of fill-ins (new non-zero
    entries) created during the elimination process.} for the factorization
  \(\matr{A} = \matr{L}\matr{D}\matr{L}\transpose{}\), \(L_\mathrm{nz}
  \coloneq\) the number of non-zeros in the \(\matr{L}\) factor.
\item exception: \(err \coloneq (\text{entries found in lower triangle} \implies
  \text{NOT\_UPPER})\)
\end{itemize}

\noindent QDLDL\_factor(\(u_f\))\footnote{A generic routine with type parameter
  \(u_f\)}(\(\matr{A}, L_\mathrm{nz}, \vec{E}\)):
\begin{itemize}
\item output: \(\matr{L}, \matr{D} \coloneq \text{factors of } \matr{A}\) in \(u_f\)
  precision, where diagonal matrix \(\matr{D}\) is simply represented by a
  vector \(\vec{d}\) as there's only non-zero elements along the diagonal.
\item exception: \(err \coloneq ((\exists\,i: \vec{d}_i = 0) \implies \text{FAC\_FAILED})\)
\end{itemize}

\noindent QDLDL\_solve(\(u_w\))(\(\matr{L}, \vec{d}, \vec{b}\)):
\begin{itemize}
\item output: solves \(\matr{A}\vec{x} = \matr{L}\matr{D}\matr{L}\transpose{} \vec{x} = \vec{b}
  \) in \(u_w\) precision, where \(\matr{D}\) is reinterpreted from \(\vec{d}\).
  Let \(\vec{y} = \matr{D}\matr{L}\transpose{} \vec{x}\), we have
  \(\matr{L}\matr{D}\matr{L}\transpose{} \vec{x} = \matr{L}
  (\matr{D}\matr{L}\transpose{} \vec{x}) = \matr{L}\vec{y}\). Solve
  \(\matr{L}\vec{y} = \vec{b}\) by back substitution (triangular solve). Solve
  \(\matr{D}\matr{L}\transpose{} \vec{x}\) by multiplying \(\vec{y}\) with
  \(\frac{1}{d_i}\) followed by another triangular solve.
\end{itemize}

\newpage

\section{MIS of Floating Point Concepts Module} \label{M:concepts}

\subsection{Module}

fp\_concepts

\subsection{Uses}

None

\subsection{Syntax}

\subsubsection{Exported Access Programs}

\begin{center}
  \begin{tabularx}{\linewidth}{%
    p{3cm}
    >{\raggedright\arraybackslash}X
    >{\raggedright\arraybackslash}l
    >{\raggedright\arraybackslash}l
    }
    \hline
    \textbf{Name}        & \textbf{In}                               & \textbf{Out}      & \textbf{Exceptions} \\
    \hline
    FloatingPoint  & tuple of \((T_1, T_2, \dots, T_n), n \geq 1\) & \(out: \mathbb{B}\) & --            \\
    PartialOrdered & tuple of \((T_1, T_2, \dots, T_n), n \geq 1\) & \(out: \mathbb{B}\) & --            \\
    Refinable      & tuple of \((T_1, T_2, \dots, T_n), n \geq 1\) & \(out: \mathbb{B}\) & --            \\
    \hline
  \end{tabularx}
\end{center}

\((T_1, T_2, \dots, T_n), n \geq 1\) is a variadic tuple of types, where \(T_n\) is some
generic type, possible values include: \texttt{int}, \texttt{double}, \texttt{float}, \dots

\subsection{Semantics}

\subsubsection{State Variables}

None

\subsubsection{Assumptions}

None

\subsubsection{Access Routine Semantics}

\noindent FloatingPoint\((T_1, T_2, \dots, T_n)\):
\begin{itemize}
\item output: \(out \coloneq (\forall\,i : \mathbb{N}\,|\,1 \leq i \leq n : T_i\;\text{is a floating point type}) \)
\end{itemize}

\noindent PartialOrdered\((T_1, T_2, \dots, T_n)\):
\begin{itemize}
\item output: \(out \coloneq (\forall\,i : \mathbb{N}\,|\,1 \leq i < n : \text{machine epsilon of} \  T_i
  > \text{machine epsilon of} \  T_{i+1}) \)
\end{itemize}

\noindent Refinable\((T_1, T_2, \dots, T_n)\):
\begin{itemize}
\item output: \(out \coloneq \text{FloatingPoint}(T_1, T_2, \dots, T_n) \land \text{PartialOrdered}(T_1, T_2, \dots, T_n)\)
\end{itemize}

\newpage

\section{MIS of Matrix Operations Module} \label{M:ops}

\subsection{Module}

ops

\subsection{Uses}

fp\_concepts (Section~\ref{M:concepts})

\subsection{Syntax}

\subsubsection{Exported Access Programs}

\begin{center}
  \renewcommand{\arraystretch}{1.2}
  \begin{tabularx}{\linewidth}{%
    >{\raggedright\arraybackslash}X
    >{\raggedright\arraybackslash}l
    >{\raggedright\arraybackslash}l
    >{\raggedright\arraybackslash}l
    }
    \hline
    \textbf{Name}                                          & \textbf{In}                           & \textbf{Out}             & \textbf{Exceptions} \\
    \hline
    MatrixMuliply(\((T, T_a, T_x)\) with \par
    Refinable\((T_a, T_x, T)\))                        & \(\matr{A}: \mathbb{R}^{n \times n}, \vec{x}: \mathbb{R}^n\) & \(\vec{res}: \mathbb{R}^n\) & --            \\
    VectorAdd(\((T, T_a, T_b)\) with \par
    Refinable\((T_a, T)\) \(\land\) Refinable\((T_b, T)\)) & \(\vec{a}: \mathbb{R}^n, \vec{b}: \mathbb{R}^n\)              & \(\vec{res}: \mathbb{R}^n\) & --            \\
    VectorSubtract(\((T, T_a, T_b)\) with \par
    Refinable\((T_a, T)\) \(\land\) Refinable\((T_b, T)\)) & \(\vec{a}: \mathbb{R}^n, \vec{b}: \mathbb{R}^n\)              & \(\vec{res}: \mathbb{R}^n\) & --            \\
    VectorMultiply(\((T, T_a, T_b)\) with \par
    Refinable\((T_a, T)\) \(\land\) Refinable\((T_b, T)\)) & \(\vec{a}: \mathbb{R}^n, \vec{b}: \mathbb{R}^n\)              & \(\vec{res}: \mathbb{R}^n\) & --            \\
    VectorScale(\((T, T_a, T_b)\) with \par
    Refinable\((T_a, T)\) \(\land\) Refinable\((T_b, T)\)) & \(\vec{a}: \mathbb{R}^n, b: \mathbb{R}\)               & \(\vec{res}: \mathbb{R}^n\) & --            \\
    VectorDot(\((T, T_a, T_b)\) with \par
    Refinable\((T_a, T)\) \(\land\) Refinable\((T_b, T)\)) & \(\vec{a}: \mathbb{R}^n, \vec{b}: \mathbb{R}^n\)              & \(\vec{res}: \mathbb{R}^n\) & --            \\
    Dnrm2(\((T, T_x)\) with Refinable\((T_x, T)\))     & \(\vec{x}: \mathbb{R}^n\)                      & \(norm: \mathbb{R}\)       & --            \\
    InfNrm(\((T, T_x)\) with Refinable\((T_x, T)\))    & \(\vec{x}: \mathbb{R}^n\)                      & \(norm: \mathbb{R}\)       & --            \\
    \hline
  \end{tabularx}
\end{center}

\subsection{Semantics}

\subsubsection{State Variables}

None

\subsubsection{Assumptions}

The (sparse) matrix used by this module is stored in CSC format.

\filbreak

\subsubsection{Access Routine Semantics}

\noindent MatrixMuliply(\((T, T_a, T_x)\) with Refinable\((T_a, T_x,
T)\))(\(\matr{A}, \vec{x}\)):
\begin{itemize}
\item output: \(\displaystyle res_i \coloneq \sum_{j = 1}^n a_{i,j} x_j\) for \(i = 1, 2,
  \dots, n\), where \(a_{i,j}\) is in precision \(T_a\), \(x_j\) is in precision
  \(T_x\), and \(res_i\) is in precision \(T\).
\end{itemize}

\noindent VectorAdd(\((T, T_a, T_b)\) with Refinable\((T_a, T)\) \(\land\)
Refinable\((T_b, T)\))(\(\vec{a}, \vec{b}\)):
\begin{itemize}
\item output: \(\displaystyle res_i \coloneq a_i + b_i\) for \(i = 1, 2, \dots, n\),
  where \(a_i\) is in precision \(T_a\), \(b_i\) is in precision \(T_b\), and
  \(res_i\) is in precision \(T\).
\end{itemize}

\noindent VectorSubtract(\((T, T_a, T_b)\) with Refinable\((T_a, T)\) \(\land\)
Refinable\((T_b, T)\))(\(\vec{a}, \vec{b}\)):
\begin{itemize}
\item output: \(\displaystyle res_i \coloneq a_i - b_i\) for \(i = 1, 2, \dots, n\),
  where \(a_i\) is in precision \(T_a\), \(b_i\) is in precision \(T_b\), and
  \(res_i\) is in precision \(T\).
\end{itemize}

\noindent VectorMultiply(\((T, T_a, T_b)\) with Refinable\((T_a, T)\) \(\land\)
Refinable\((T_b, T)\))(\(\vec{a}, \vec{b}\)):
\begin{itemize}
\item output: \(\displaystyle res_i \coloneq a_i b_i\) for \(i = 1, 2, \dots, n\), where
  \(a_i\) is in precision \(T_a\), \(b_i\) is in precision \(T_b\), and
  \(res_i\) is in precision \(T\).
\end{itemize}

\noindent VectorScale(\((T, T_a, T_b)\) with Refinable\((T_a, T)\) \(\land\)
Refinable\((T_b, T)\))(\(\vec{a}, b\)):
\begin{itemize}
\item output: \(\displaystyle res_i \coloneq b a_i\) for \(i = 1, 2, \dots, n\), where
  \(a_i\) is in precision \(T_a\), \(b\) is in precision \(T_b\), and \(res_i\)
  is in precision \(T\).
\end{itemize}

\noindent VectorDot(\((T, T_a, T_b)\) with Refinable\((T_a, T)\) \(\land\)
Refinable\((T_b, T)\))(\(\vec{a}, \vec{b}\)):
\begin{itemize}
\item output: \(\displaystyle res \coloneq \sum_{i=1}^n a_ib_i\), where \(a_i\) is in
  precision \(T_a\), \(b_i\) is in precision \(T_b\), and \(res\) is in
  precision \(T\).
\end{itemize}

\noindent Dnrm2(\((T, T_x)\) with Refinable\((T_x, T)\))(\(\vec{x}\)):
\begin{itemize}
\item output: \(\displaystyle norm \coloneq \sqrt{\sum_{i=1}^n x_i^2}\), where \(x_i\) is in
  precision \(T_x\), and \(norm\) is in precision \(T\).
\end{itemize}

\noindent InfNrm(\((T, T_x)\) with Refinable\((T_x, T)\))(\(\vec{x}\)):
\begin{itemize}
\item output: \(\displaystyle norm \coloneq \max_{1 \leq i \leq n}|x_i|\), where \(x_i\) is
  in precision \(T_x\), and \(norm\) is in precision \(T\).
\end{itemize}

\newpage

\section{MIS of Iterative Solver Module} \label{M:solve}

\subsection{Module}

GmresLDLIR\((u_f, u_w, u_r)\) with Refinable\((u_f, u_w, u_r)\)

\subsection{Uses}

qdldl (Section~\ref{M:factor}), fp\_concepts (Section~\ref{M:concepts}), ops
(Section~\ref{M:ops})

\subsection{Syntax}

\subsubsection{Exported Access Programs}

\begin{center}
  \begin{tabularx}{\linewidth}{%
    p{5cm}
    >{\raggedright\arraybackslash}X
    >{\raggedright\arraybackslash}X
    >{\raggedright\arraybackslash}l
    }
    \hline
    \textbf{Name}               & \textbf{In}                   & \textbf{Out}     & \textbf{Exceptions} \\
    \hline
    Compute               & \(\matr{A}: \mathbb{R}^{n \times n}\) & --         & --            \\
    Solve                 & \(\vec{b}: \mathbb{R}^n\)              & \(\vec{x}: \mathbb{R}^n\) & --            \\
    SetMaxIRIterations    & \(n: \mathbb{N}\)               & --         & --            \\
    SetMaxGmresIterations & \(n: \mathbb{N}\)               & --         & --            \\
    SetTolerance          & \(\epsilon: \mathbb{R}\)               & --         & --            \\
    \hline
  \end{tabularx}
\end{center}

\subsection{Semantics}

\subsubsection{State Variables}

\begin{itemize}[noitemsep,topsep=0pt,leftmargin=2mm,label={}]
\item n\_: \(\mathbb{N}\), size of the matrix \(\matr{A}\)
\item Ap\_: \(\mathbb{N}^n\), column pointers of \(\matr{A}\) (part of CSC format)
\item Ai\_: \(\mathbb{N}^{n_\mathrm{nz}}\), row indices of \(\matr{A}\) (part of CSC format),
  \(n_\mathrm{nz}\) is the number of non-zeros
\item Ax\_: \(\mathbb{R}^{n_\mathrm{nz}}\), non-zero values of \(\matr{A}\) (part of CSC format),
  in precision \(u_w\)
\item Lp\_: \(\mathbb{N}^n\), column pointers of the \(\matr{L}\) factor
\item Li\_: \(\mathbb{N}^{L_\mathrm{nz}}\), row indices of the \(\matr{L}\) factor
\item Lx\_: \(\mathbb{R}^{L_\mathrm{nz}}\), non-zero values of the \(\matr{L}\) factor in precision
  \(u_f\)
\item Dinv\_: \(\mathbb{R}^n\), non-zero values of the inverse of the \(\matr{D}\)
  factor in precision \(u_f\)
\item ir\_iter\_: \(\mathbb{N}\), maximum refinement iterations, default to 10
\item gmres\_iter\_: \(\mathbb{N}\), maximum GMRES iterations, default to 10
\item tol\_: \(\mathbb{R}\), tolerance in precision \(u_r\), default to \num{1e-10}
\end{itemize}

\subsubsection{Assumptions}

The input matrix \(\matr{A}\) is non-singular, symmetric quasi-definite, and is
stored in CSC format.

\subsubsection{Access Routine Semantics}


\noindent SetMaxIRIterations(\(n\)):
\begin{itemize}
\item transition: ir\_iter\_ \(\coloneq n\)
\end{itemize}

\noindent SetMaxGmresIterations(\(n\)):
\begin{itemize}
\item transition: gmres\_iter\_ \(\coloneq n\)
\end{itemize}

\noindent SetTolerance(\(\epsilon\)):
\begin{itemize}
\item transition: tol\_ \(\coloneq \epsilon\)
\end{itemize}

\noindent Compute(\(\matr{A}\)):
\begin{itemize}
\item transition: \par
  n\_ \(\coloneq\) size of \(\matr{A}\), \par
  Ap\_, Ai\_, Ax\_ \(\coloneq \matr{A}\), \par
  Lp\_, Li\_, Lx\_, 1 / Dinv\_ \(\coloneq \text{QDLDL\_factor}(u_f)(\matr{A}, L_\mathrm{nz}, \vec{E})\), \par
  where \(L_\mathrm{nz}, \vec{E} \coloneq \text{QDLDL\_etree}(u_f)(\matr{A})\)

\end{itemize}

\noindent Solve(\(\vec{b}\)):
\begin{itemize}
\item output: solves \(\matr{A}\vec{x} = \vec{b} \) in \(u_w\) precision following
  Algorithm~\ref{algo:gmresir}.
\end{itemize}

\subsubsection{Algorithms}

These are copied from \iref{IM:IM} in the SRS (\cite{SRS}).
\begin{algorithm}[H]
  \caption{GMRES-IR with \(\matr{L}\matr{D}\matr{L}\transpose\) factorization in MP}
  \label{algo:gmresir}
  \begin{algorithmic}[1]
    \State Perform \(\matr{L}\matr{D}\matr{L}\transpose\) factorization of \(\matr{A}\) \Comment{at \(u_f\)}
    \State Solve \(\matr{L}\matr{D}\matr{L}\transpose \vec{x}_0 = \vec{b}\) \Comment{at \(u_f\)}
    \For{\(i \gets 0, 1, \dots, n_\mathrm{iter}\) and \(\norm{r_i}_2 \geq \epsilon\)}
      \State \(r_i \gets \vec{b} - \matr{A}\vec{x}_i\) \Comment{at \(u_r\)}

      \State Solve \((\matr{L}\matr{D}\matr{L}\transpose)^{-1}\matr{A}\vec{d}_i =
      (\matr{L}\matr{D}\matr{L}\transpose)^{-1}\vec{r_i}\) with GMRES (See
      Algorithm~\ref{algo:gmres}) \par
      where \(\matr{M}^{-1} = (\matr{L}\matr{D}\matr{L}\transpose)^{-1}\) \Comment{at \(u_w\)}
      \State \(\vec{x}_{i + 1} = \vec{x}_i + \vec{d}_i\) \Comment{at \(u_w\)}
    \EndFor
  \end{algorithmic}
\end{algorithm}

\begin{algorithm}[H]
  \caption{Restarted GMRES with left preconditioning}
  \label{algo:gmres}
  \begin{algorithmic}[1]
    \State \(\matr{A} \in \mathbb{R}^{n \times n}, \quad \vec{x}_0, \vec{b} \in \mathbb{R}^n, \quad \matr{M}^{-1} \approx \matr{A}^{-1}\)
    \For{\(k \gets 1, 2, \dots\), the \(k\)th restart}
      \State \(\vec{z}_k \gets \vec{b} - \matr{A}\vec{x}_k\) \Comment{Compute residual}
      \State \(\vec{r}_k \gets \matr{M}^{-1}\vec{z}_k\) \Comment{Apply preconditioning}
      \State \(\beta \gets \norm{\vec{r}_k}_2, \quad \vec{v}_1 = \vec{r}_k / \beta, \quad \matr{V}_1 \gets [\vec{v}_1]\) \Comment{Setup for Arnoldi process}
      \State Construct orthogonal basis of preconditioned Krylov subspace \[\mathcal{K}_m(\matr{M}^{-1}\matr{A}, \vec{r}_k) = \spann{\vec{r}_k, \matr{M}^{-1}\matr{A}\vec{r}_k, \dots, (\matr{M}^{-1}\matr{A})^{m - 1}\vec{r}_k}\]
      \State The Arnoldi process gives \[\matr{M}^{-1}\matr{A}\matr{V}_m = \matr{V}_{m+1}\bar{\matr{H}}_m\]
      where \(\matr{V}_m \in \mathbb{R}^{n \times m}\) contains orthonormal basis vectors for \(\mathcal{K}_m\),
      \(\matr{V}_{m+1} \in \mathbb{R}^{n \times (m+1)}\) extends \(\matr{V}_m\) with one more vector,
      \(\bar{\matr{H}}_m \in \mathbb{R}^{(m+1) \times m}\) is an upper Hessenberg matrix
      \State Solve the least square problem \[\min_{\vec{y}_m\in\mathbb{R}^m} \norm{\beta e_1 - \bar{\matr{H}}_m \vec{y}_m}_2\]
      where \(e_1 \in \mathbb{R}^{m + 1}\) is the first standard basis vector \(e_1 = [1, 0, 0, \dots, 0]\transpose\)
      \State \(\vec{x}_{k + 1} = \vec{x}_k + \matr{V}_m \vec{y}_m\) \Comment{Add the correction}
    \EndFor
  \end{algorithmic}
\end{algorithm}

\wss{A module without environment variables or state variables is unlikely to
  have a state transition.  In this case a state transition can only occur if
  the module is changing the state of another module.}

\wss{Modules rarely have both a transition and an output.  In most cases you
  will have one or the other.}

% \subsubsection{Local Functions}

\wss{As appropriate} \wss{These functions are for the purpose of specification.
  They are not necessarily something that is going to be implemented
  explicitly.  Even if they are implemented, they are not exported; they only
  have local scope.}

\newpage

\printbibliography{}

\end{document}
