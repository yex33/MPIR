\documentclass[12pt, titlepage]{article}

\usepackage{amsmath, mathtools}

\usepackage{amsfonts}
\usepackage{amssymb}
\usepackage{graphicx}
\usepackage{colortbl}
\usepackage{xr,xr-hyper}
\usepackage{hyperref}
\usepackage{longtable}
\usepackage{xfrac}
\usepackage{tabularx}
\usepackage{float}
\usepackage{siunitx}
\usepackage{booktabs}
\usepackage{multirow}
\usepackage[section]{placeins}
\usepackage{caption}
\usepackage{fullpage}

\hypersetup{
  bookmarks=true,    % show bookmarks bar?
  colorlinks=true,   % false: boxed links; true: colored links
  linkcolor=red,     % color of internal links (change box color with linkbordercolor)
  citecolor=blue,    % color of links to bibliography
  filecolor=magenta, % color of file links
  urlcolor=cyan      % color of external links
}

\usepackage{array}

\externaldocument{../../SRS/SRS}
\newcommand{\iref}[1]{IM\ref{#1}}

\input{../../Comments}
%% Common Parts

\newcommand{\progname}{MPIR} % PUT YOUR PROGRAM NAME HERE
\newcommand{\authname}{Xunzhou (Joe) Ye} % AUTHOR NAMES

\newcommand{\matr}[1]{\mathbf{#1}}
\renewcommand{\vec}[1]{\mathbf{#1}}
\newcommand{\spann}[1]{\mathrm{span}\{#1\}}

\usepackage{hyperref}
    \hypersetup{colorlinks=true, linkcolor=blue, citecolor=blue, filecolor=blue,
                urlcolor=blue, unicode=false}
    \urlstyle{same}


\usepackage[backend=biber,style=authoryear]{biblatex}
\addbibresource{../../../refs/References.bib}

\begin{document}

\title{Module Interface Specification for \progname{}}

\author{\authname}

\date{\today}

\maketitle

\pagenumbering{roman}

\section{Revision History}

\begin{tabularx}{\textwidth}{p{3cm}p{2cm}X}
\toprule {\bf Date} & {\bf Version} & {\bf Notes}\\
\midrule
\date{19 March 2025} & 1.0 & Initial draft \\
\bottomrule
\end{tabularx}

~\newpage

\section{Symbols, Abbreviations and Acronyms}

See SRS Documentation at \cite{SRS}

\newpage

\tableofcontents

\newpage

\pagenumbering{arabic}

\section{Introduction}

The following document details the Module Interface Specifications for
\progname{}. It is intended to solve a sparse linear system with an iterative
method in mixed-precisions.

Complementary documents include the System Requirement Specifications and Module
Guide. The full documentation and implementation can be found at \cite{SRS} and
\cite{MG}.

\section{Notation}

\wss{You should describe your notation.  You can use what is below as
  a starting point.}

The structure of the MIS for modules comes from \cite{HoffmanAndStrooper1995}, with
the addition that template modules have been adapted from \cite{GhezziEtAl2003}.
The mathematical notation comes from Chapter 3 of \cite{HoffmanAndStrooper1995}.
For instance, the symbol := is used for a multiple assignment statement and
conditional rules follow the form
$(c_1 \Rightarrow r_1 | c_2 \Rightarrow r_2 | ... | c_n \Rightarrow r_n )$.

The following table summarizes the primitive data types used by \progname.

\begin{center}
  \renewcommand{\arraystretch}{1.2}
  \noindent
  \begin{tabular}{l l p{7.5cm}}
    \toprule
    \textbf{Data Type}   & \textbf{Notation} & \textbf{Description}                                         \\
    \midrule
    character      & char        & a single symbol or digit                               \\
    integer        & $\mathbb{Z}$        & a number without a fractional component in (-$\infty$, $\infty$) \\
    natural number & $\mathbb{N}$        & a number without a fractional component in [1, $\infty$)    \\
    real           & $\mathbb{R}$        & any number in (-$\infty$, $\infty$)                              \\
    \bottomrule
  \end{tabular}
\end{center}

\noindent
The specification of \progname \ uses some derived data types: sequences, strings, and
tuples. Sequences are lists filled with elements of the same data type. Strings
are sequences of characters. Tuples contain a list of values, potentially of
different types. In addition, \progname \ uses functions, which
are defined by the data types of their inputs and outputs. Local functions are
described by giving their type signature followed by their specification.

\section{Module Decomposition}

The following table is taken directly from the Module Guide document for this project.

\begin{table}[H]
  \centering
  \begin{tabular}{p{0.3\textwidth} p{0.6\textwidth}}
    \toprule
    \textbf{Level 1}                                            & \textbf{Level 2}                               \\
    \midrule
    {Hardware-Hiding Module}                              & --                                       \\
    \midrule
    \multirow{4}{0.3\textwidth}{Behaviour-Hiding Module}  & Read Matrix Module                       \\
                                                          & Solver Output Module                     \\
                                                          & Matrix Factorization Module              \\
                                                          & Iterative Solves Module                  \\
    \midrule
    \multirow{1}{0.3\textwidth}{Software Decision Module} & Example Program Module                   \\
    \bottomrule
  \end{tabular}
  \caption{Module Hierarchy}
  \label{TblMH}
\end{table}

\newpage

\section{MIS of Matrix Factorization Module} \label{M:factor}

\subsection{Module}

Factorize

\subsection{Uses}

None

\subsection{Syntax}

\subsubsection{Exported Access Programs}

\begin{center}
  \begin{tabular}{p{3cm} p{5cm} p{5cm} p{2cm}}
    \hline
    \textbf{Name}       & \textbf{In}                                           & \textbf{Out}                          & \textbf{Exceptions} \\
    \hline
    QDLDL\_etree  & \(\matr{A}: \mathbb{R}^{n \times n}\)                         & \(L_\mathrm{nz}: \mathbb{N}, \vec{E}: \mathbb{R}^n\)          & NOT\_UPPER    \\
    QDLDL\_factor & \(\matr{A}: \mathbb{R}^{n \times n}, L_\mathrm{nz}: \mathbb{N}, \vec{E}: \mathbb{R}^n, u_f\) & \(\matr{L}: \mathbb{R}^{n \times n}, \vec{d}: \mathbb{R}^n\) & FAC\_FAILED   \\
    QDLDL\_solve  & \(\matr{L}: \mathbb{R}^{n \times n}, \vec{d}: \mathbb{R}^n, \vec{b}: \mathbb{R}^n, u_w\)     & \(\vec{x}: \mathbb{R}^n\)                      & --            \\
    \hline
  \end{tabular}
\end{center}

\subsection{Semantics}

\subsubsection{State Variables}

None

\subsubsection{Assumptions}

The (sparse) matrix used or returned by this module is stored in Compressed
Sparse Column (CSC) format. The input matrix \(\matr{A}\) only contains data for
the upper triangular part.

\subsubsection{Access Routine Semantics}

\noindent QDLDL\_etree(\(\matr{A}\)):
\begin{itemize}
\item output: \(\vec{E} \coloneq\) elimination tree for the factorization \(\matr{A} =
  \matr{L}\matr{D}\matr{L}\transpose{}\), \(L_\mathrm{nz} \coloneq\) the number of
  non-zeros in the \(\matr{L}\) factor.
\item exception: \(err \coloneq (\text{entries found in lower triangle} \implies
  \text{NOT\_UPPER})\)
\end{itemize}

\noindent QDLDL\_factor<\(u_f\)>(\(\matr{A}, L_\mathrm{nz}, \vec{E}\)):
\begin{itemize}
\item output: \(\matr{L}, \matr{D} \coloneq \text{factors of } \matr{A}\) in \(u_f\)
  precision, where diagonal matrix \(\matr{D}\) is simply represented by a
  vector \(\vec{d}\) as there's only non-zero elements along the diagonal.
\item exception: \(err \coloneq ((\exists\,i: \vec{d}_i = 0) \implies \text{FAC\_FAILED})\)
\end{itemize}

\noindent QDLDL\_solve<\(u_w\)>(\(\matr{L}, \vec{d}, \vec{b}\)):
\begin{itemize}
\item output: solves \(\matr{A}\vec{x} = \matr{L}\matr{D}\matr{L}\transpose{} \vec{x} = \vec{b}
  \) in \(u_w\) precision, where \(\matr{D}\) is reinterpreted from \(\vec{d}\).
  Let \(\vec{y} = \matr{D}\matr{L}\transpose{} \vec{x}\), we have
  \(\matr{L}\matr{D}\matr{L}\transpose{} \vec{x} = \matr{L}
  (\matr{D}\matr{L}\transpose{} \vec{x}) = \matr{L}\vec{y}\). Solve
  \(\matr{L}\vec{y} = \vec{b}\) by back substitution (triangular solve). Solve
  \(\matr{D}\matr{L}\transpose{} \vec{x}\) by multiplying \(\vec{y}\) with
  \(\frac{1}{d_i}\) followed by another triangular solve.
\end{itemize}

\newpage

\section{MIS of Solver Output Module} \label{M:output}

\subsection{Module}

Stats

\subsection{Uses}

None


\subsection{Semantics}

\subsubsection{State Variables}

\begin{center}
  \begin{tabular}{p{2cm} p{8cm}}
    \hline
    \textbf{Field}           & \textbf{Description}                                 \\
    \hline
    \(n: \mathbb{N}\)          & size of the matrix being solved                \\
    \(n_\mathrm{nz}: \mathbb{N}\)     & number of non-zeros in the matrix              \\
    \(L_\mathrm{nz}: \mathbb{N}\)     & number of non-zeros in the \(\matr{L}\) factor \\
    \(t_\mathrm{fact}: \mathbb{R}\)   & time spent in numeric factorization            \\
    \(t_\mathrm{solve}: \mathbb{R}\)  & time spent in iterative solves                 \\
    \(n_\mathrm{refine}: \mathbb{N}\) & number of refinement steps                     \\
    \hline
  \end{tabular}
\end{center}

\newpage

\section{MIS of Iterative Solves Module} \label{M:solve}

\subsection{Module}

Solve

\subsection{Uses}

Factorize (Section~\ref{M:factor}), Stats (Section~\ref{M:output})

\subsection{Syntax}

\subsubsection{Exported Access Programs}

\begin{center}
  \begin{tabular}{p{2cm} p{8cm} p{3cm} p{2cm}}
    \hline
    \textbf{Name} & \textbf{In}                                                            & \textbf{Out}               & \textbf{Exceptions} \\
    \hline
    solve   & \(\matr{A}: \mathbb{R}^{n \times n}, \vec{b}: \mathbb{R}^n, \epsilon: \mathbb{R}, n_\mathrm{iter}: \mathbb{N}, u_f, u_w, u_r\) & \(\vec{x}: \mathbb{R}^n\), s: Stats & --            \\
    \hline
  \end{tabular}
\end{center}

\subsection{Semantics}

\subsubsection{Assumptions}

The input matrix \(\matr{A}\) is non-singular, quasi-definite, and is stored in
CSC format.

\subsubsection{Access Routine Semantics}

\noindent solve<\(u_f, u_w, u_r\)>(\(\matr{A}, \vec{b}, \epsilon, n_\mathrm{iter}\)):
\begin{itemize}
\item transition: fill all fields in an instance of Stats record.
\item output: solves \(\matr{A}\vec{x} = \vec{b} \) in \(u_w\) precision following the algorithm
  specified in \iref{IM:IM} in \cite{SRS}.
\end{itemize}

\newpage

\section{MIS of Read Matrix Module} \label{M:input}

\subsection{Module}

ReadMatrix

\subsection{Uses}

None

\subsection{Syntax}

\subsubsection{Exported Access Programs}

\begin{center}
  \begin{tabular}{p{3cm} p{5cm} p{5cm} p{2cm}}
    \hline
    \textbf{Name}     & \textbf{In}            & \textbf{Out}                  & \textbf{Exceptions} \\
    \hline
    ReadMMtoCSC & filename: string & \(\matr{A}: \mathbb{R}^{n \times n}\) & --            \\
    \hline
  \end{tabular}
\end{center}

\subsection{Semantics}

\subsubsection{Assumptions}

The file (filename) contains a sparse matrix in Matrix Market Exchange format
(\cite{noauthor_matrix_2013}).

\subsubsection{Access Routine Semantics}

\noindent ReadMMtoCSC(filename):
\begin{itemize}
\item output: \(\matr{A} \coloneq \) the sparse matrix parsed and compressed in CSC.
\end{itemize}

\newpage

\section{MIS of Example Program Module} \label{M:example}

\subsection{Module}

Example

\subsection{Uses}

ReadMatrix (Section~\ref{M:input}), Solve (Section~\ref{M:solve}), Stats (Section~\ref{M:input})

\subsection{Syntax}

\subsubsection{Exported Access Programs}

\begin{center}
  \begin{tabular}{p{3cm} p{5cm} p{5cm} p{2cm}}
    \hline
    \textbf{Name} & \textbf{In}            & \textbf{Out} & \textbf{Exceptions} \\
    \hline
    demo    & filename: string & -      & -             \\
    \hline
  \end{tabular}
\end{center}

\subsection{Semantics}

\subsubsection{Access Routine Semantics}

\noindent demo():
\begin{itemize}
\item transition: read matrix from file (filename), invoke the solver with a
  predefined set of \(\vec{b}, \epsilon, n_\mathrm{iter}, u_f, u_w, u_r\), and
  finally print the outputs and stats returned by the solver.
\end{itemize}

\wss{A module without environment variables or state variables is unlikely to
  have a state transition.  In this case a state transition can only occur if
  the module is changing the state of another module.}

\wss{Modules rarely have both a transition and an output.  In most cases you
  will have one or the other.}

% \subsubsection{Local Functions}

\wss{As appropriate} \wss{These functions are for the purpose of specification.
  They are not necessarily something that is going to be implemented
  explicitly.  Even if they are implemented, they are not exported; they only
  have local scope.}

\newpage

\printbibliography{}

\end{document}
