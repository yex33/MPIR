\documentclass{article}

\usepackage{tabularx}
\usepackage{booktabs}

\title{Reflection and Traceability Report on \progname}

\author{\authname}

\date{}

\input{../Comments}
%% Common Parts

\newcommand{\progname}{MPIR} % PUT YOUR PROGRAM NAME HERE
\newcommand{\authname}{Xunzhou (Joe) Ye} % AUTHOR NAMES

\newcommand{\matr}[1]{\mathbf{#1}}
\renewcommand{\vec}[1]{\mathbf{#1}}
\newcommand{\spann}[1]{\mathrm{span}\{#1\}}

\usepackage{hyperref}
    \hypersetup{colorlinks=true, linkcolor=blue, citecolor=blue, filecolor=blue,
                urlcolor=blue, unicode=false}
    \urlstyle{same}


\begin{document}

\maketitle

\plt{Reflection is an important component of getting the full benefits from a
learning experience.  Besides the intrinsic benefits of reflection, this
document will be used to help the TAs grade how well your team responded to
feedback.  Therefore, traceability between Revision 0 and Revision 1 is and
important part of the reflection exercise.  In addition, several CEAB (Canadian
Engineering Accreditation Board) Learning Outcomes (LOs) will be assessed based
on your reflections.}

\section{Changes in Response to Feedback}

\plt{Summarize the changes made over the course of the project in response to
feedback from TAs, the instructor, teammates, other teams, the project
supervisor (if present), and from user testers.}

\plt{For those teams with an external supervisor, please highlight how the feedback
from the supervisor shaped your project.  In particular, you should highlight the
supervisor's response to your Rev 0 demonstration to them.}

\plt{Version control can make the summary relatively easy, if you used issues
and meaningful commits.  If you feedback is in an issue, and you responded in
the issue tracker, you can point to the issue as part of explaining your
changes.  If addressing the issue required changes to code or documentation, you
can point to the specific commit that made the changes.  Although the links are
helpful for the details, you should include a label for each item of feedback so
that the reader has an idea of what each item is about without the need to click
on everything to find out.}

\plt{If you were not organized with your commits, traceability between feedback
and commits will not be feasible to capture after the fact.  You will instead
need to spend time writing down a summary of the changes made in response to
each item of feedback.}

\plt{You should address EVERY item of feedback.  A table or itemized list is
recommended.  You should record every item of feedback, along with the source of
that feedback and the change you made in response to that feedback.  The
response can be a change to your documentation, code, or development process.
The response can also be the reason why no changes were made in response to the
feedback.  To make this information manageable, you will record the feedback and
response separately for each deliverable in the sections that follow.}

\plt{If the feedback is general or incomplete, the TA (or instructor) will not
be able to grade your response to feedback.  In that case your grade on this
document, and likely the Revision 1 versions of the other documents will be
low.}

All feedback Github issues were closed with a comment. It either provides a link
to the commit addressing the issue or explains why the issue was rejected.
Please refer to
\href{https://github.com/yex33/MPIR/issues?q=is\%3Aissue\%20state\%3Aclosed}{Closed
  Github issues} for tracability.

\subsection{SRS and Hazard Analysis}

\subsection{Design and Design Documentation}

\subsection{VnV Plan and Report}

\section{Challenge Level and Extras}

\subsection{Challenge Level}

\plt{State the challenge level (advanced, general, basic) for your project. Your
  challenge level should exactly match what is included in your problem
  statement. This should be the challenge level agreed on between you and the
  course instructor.}

Copied from the problem statement:

\begin{quote}
  The challenge level of this project is expected to be general, leaning toward
  the advanced level for the following reasons:
  \begin{enumerate}
  \item This is a continuation of a Master's level research project. The
    associated thesis is still working in progress.
  \item The choice of algorithm and mathematical technique is predefined as part
    of the project. The mathematical knowledge required for understanding and
    implementing these algorithms (e.g. GMRES) is graduate level.
  \item Working with multiple floating point precisions requires knowledge on
    computer architectures, low-level floating point representations in
    computer, as well as advance implementation techniques to support
    manipulations of precisions.
  \item Testing, benchmarking, and evaluating the performance gain on applying
    these advanced algorithms and techniques are also very involved.
  \end{enumerate}
\end{quote}

\subsection{Extras}

\plt{Summarize the extras (if any) that were tackled by this project.  Extras
can include usability testing, code walkthroughs, user documentation, formal
proof, GenderMag personas, Design Thinking, etc.  Extras should have already
been approved by the course instructor as included in your problem statement.}

Usability testing. Please refer to the
\href{https://github.com/yex33/MPIR/blob/main/docs/VnVReport/VnVReport.pdf}{VnVReport}
for a detailed report.

\section{Design Iteration (LO11 (PrototypeIterate))}

\plt{Explain how you arrived at your final design and implementation.  How did
the design evolve from the first version to the final version?}

\plt{Don't just say what you changed, say why you changed it.  The needs of the
client should be part of the explanation.  For example, if you made changes in
response to usability testing, explain what the testing found and what changes
it led to.}

The final design and implementation of \progname{} evolved significantly from
the initial prototype, guided by feedback, usability insights, and practical
constraints encountered during development. The initial version focused on
quickly prototyping the core numerical algorithm—preconditioned GMRES with
iterative refinement—without strong emphasis on modularity, extensibility, or
type safety.

As development progressed, a key insight was the need to reduce the complexity
of the interface. Early versions exposed many low-level numerical and type
parameters, which were error-prone and hard to use correctly. In response, later
iterations introduced the \texttt{Refinable} concept to encapsulate type constraints
and reduce redundant template parameters. This abstraction simplified the solver
API while maintaining flexibility across mixed-precision configurations.

Usability testing with internal team members highlighted pain points in
documentation clarity and error messages. Based on this, improvements were made
to ensure better self-documentation of function names, improved error reporting,
and clearer usage examples. The solver interface was also made more idiomatic by
replacing raw pointer manipulation with standard containers such as
\texttt{std::vector}, greatly improving readability and safety.

Performance benchmarking also shaped the final implementation. Manual timing and
profiling helped restructure certain steps (e.g., Gram-Schmidt and Givens
rotations) into modular helpers, making them easier to optimize and validate
independently.

While the project did not undergo multiple full development cycles due to time
constraints, targeted refinements were made throughout the process based on
stakeholder feedback and empirical testing.

\section{Design Decisions (LO12)}

\plt{Reflect and justify your design decisions.  How did limitations,
 assumptions, and constraints influence your decisions?  Discuss each of these
 separately.}

The design of \progname{} was shaped by a combination of research goals,
technical constraints, and assumptions inherited from prior work.

\begin{description}
\item[Limitations] Time and manpower limitations constrained the scope of
  the features implemented. For example, although support for multiple sparse
  matrix formats was considered, only Compressed Sparse Column (CSC) was
  supported in the final release. Similarly, maintainability testing and full
  documentation coverage were deprioritized in favor of ensuring correctness and
  performance.
\item[Assumptions] The project assumed that the input matrices would be
  symmetric and quasi-definite, a constraint imposed by the use of QDLDL as the
  preconditioner. It also assumed that users of the library are domain experts,
  which influenced decisions around error handling and user interface
  simplicity.
\item[Constraints] The software was designed to interface with a previously
  developed research prototype, which constrained design flexibility. Adhering
  to this compatibility requirement meant retaining certain data structures and
  algorithmic sequences, even when alternatives might have offered better
  abstraction or performance.
\end{description}

Overall, design decisions were made to balance innovation with alignment to the
existing research direction, prioritizing numerical correctness, performance
validation, and software quality within the defined constraints.

\section{Economic Considerations (LO23)}

\plt{Is there a market for your product? What would be involved in marketing your
product? What is your estimate of the cost to produce a version that you could
sell?  What would you charge for your product?  How many units would you have to
sell to make money? If your product isn't something that would be sold, like an
open source project, how would you go about attracting users?  How many potential
users currently exist?}

\progname{} is intended as a research-grade open-source numerical library rather
than a commercial software product. As such, its primary value lies in academic
and scientific use, particularly for researchers working with sparse linear
systems and mixed-precision methods.

There is a niche but growing community in computational science, optimization,
and numerical linear algebra that could benefit from such a solver—particularly
in the context of accelerating computation with mixed-precision techniques.
Potential users include graduate researchers, academic labs, and developers of
scientific computing software.

To attract users, the focus would be on clear documentation, published
benchmarks, and integration examples that demonstrate its advantages over
existing libraries. Publishing results in a peer-reviewed venue and promoting
through academic mailing lists and GitHub repositories would be essential for
adoption.

If commercialized, the costs would include packaging, support, performance
tuning, and robust cross-platform support. However, due to its specialized
nature and open-source orientation, monetization is unlikely to be the focus.
Instead, success would be measured by citations, usage in downstream research,
and community contributions.


\section{Reflection on Project Management (LO24)}

\plt{This question focuses on processes and tools used for project management.}

\subsection{How Does Your Project Management Compare to Your Development Plan}

\plt{Did you follow your Development plan, with respect to the team meeting plan,
team communication plan, team member roles and workflow plan.  Did you use the
technology you planned on using?}

At the outset, there was no formal development plan. As a solo project,
responsibilities such as implementation, documentation, and testing were all
managed independently and informally. As the project matured, informal task
lists and GitHub issues were introduced to organize priorities, which improved
the workflow during later phases.

\subsection{What Went Well?}

\plt{What went well for your project management in terms of processes and
technology?}

The use of GitHub for version control, issue tracking, and continuous
integration was particularly effective. Lightweight tools such as CMake,
clang-format, and clang-tidy helped enforce code quality without introducing too
much overhead. Despite the absence of a team, consistent incremental development
and documentation practices enabled steady progress and maintainability.

\subsection{What Went Wrong?}

\plt{What went wrong in terms of processes and technology?}

Without a structured development plan from the beginning, certain
tasks—especially verification planning, usability testing, and code cleanup—were
deferred until late in the timeline. This led to some last-minute work and
limited the opportunity for multiple design iterations. The solo nature of the
project also meant that there was limited opportunity for collaborative feedback
or delegation of tasks, which might have improved efficiency.

\subsection{What Would you Do Differently Next Time?}

\plt{What will you do differently for your next project?}

For future projects, even solo ones, creating a lightweight but structured
development plan early on would help manage scope and track progress more
effectively. This plan would include estimated timelines, checkpoints, and
clearer criteria for success. Introducing formal retrospectives or progress
reviews at regular intervals would also provide opportunities to reflect and
adjust the approach as needed.

\section{Reflection on Capstone}

Not applicable for CAS 741

\plt{This question focuses on what you learned during the course of the capstone project.}

\subsection{Which Courses Were Relevant}

N/A

\plt{Which of the courses you have taken were relevant for the capstone project?}

\subsection{Knowledge/Skills Outside of Courses}

N/A

\plt{What skills/knowledge did you need to acquire for your capstone project
that was outside of the courses you took?}

\end{document}
